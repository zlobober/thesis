The facts above are enough to provide and prove the original algorithm \ref{alg:n8} of Eilam-Tzoreff.

Let us estimate the running time of all parts of the algorithm.

\begin{proposition}
Construction of a pairwise distance matrix in line \ref{line:floyd} of algorithm \ref{alg:n8} may be done in $\OO(|V|^3)$.
\end{proposition}
\begin{proof}
We can use a well-known algorithm of Floyd-Warshall (\cite{Floyd}, \cite{CLRS}).
\end{proof}

\begin{proposition}
Procedure \textsc{CalculateSingle2DSPValue} works in $\OO(1)$, $\OO(|V|)$ or $\OO(|V|^4)$ depending on which case of theorem \ref{main_theorem} happens.
\end{proposition}
\begin{proof}
Line \ref{line:a} requires constant time, line \ref{line:b} corresponds to a case in which the result depends on $\OO(|V|)$ quadruples (cf. part 2 of theorem \ref{main_theorem}). 

Line \ref{line:cQ2} corresponds to a group of dependent quadruples that differ from $(s_1, t_1, s_2, t_2)$ in two components, hence, this line requires no more than $\OO(|V|^2)$ operations.

Finally, line \ref{line:cQ4} corresponds to a group of dependent quadruples that differ from $(s_1, t_1, s_2, t_2)$ in four components, hence, this line requires no more than $\OO(|V|^4)$ operations. \end{proof}

It is easy to see that the most inefficient part of a program is calculation of $Q_4(s_1, t_1, s_2, t_2)$. Let us show that the algorithm \ref{alg:n8} may actually work in $\Theta(|V|^8)$ on specific graphs. In order to do so, we should present a family of arbitrary large graphs for which the running time of algorithm \ref{alg:n8} will be arbitrarily large. 

\begin{theorem}
Let's build a graph $G_k = (V_k, E_k)$, in which

\begin{align}
V_k &= \bigsqcup\limits_{i = 0,\ldots,7} S_i \sqcup \{c\} \\
E_k &= \bigsqcup\limits_{i = 0,\ldots,7} S_i \times S_{(i+1) \bmod 8} \quad\sqcup\quad \bigsqcup\limits_{i=1,3,5,7} S_i \times \{c\} \\
|S_i| &= k.
\end{align}

\begin{figure}[H]
\caption{Graph $G_k$, for which running time of algorithm \ref{alg:n8} is $\Theta(|V|^8)$}
\centering
\includegraphics[width=0.5\textwidth]{pic.1}
\end{figure}

Note that for the constructed graph $|V_k| = 8k + 1$ and $|E_k| = 8k^2 + 4k$. The running time of the algorithm \ref{alg:n8} at graph $G_k$ is $\Theta(k^8) = \Theta(|V|^8)$.

\end{theorem}
\begin{proof}
Suppose that $s_1 \in S_0$, $s_2 \in S_2$, $t_1 \in S_4$, $t_2 \in S_6$. In this case $l(s_1, t_1) = l(s_2, t_2) = 4$, $l(s_1, s_2) = l(s_2, t_1) = l(t_1, t_2) = l(t_2, s_1) = 2$, i.e. $s_1, t_1 \in L(s_2, t_2)$ i $s_2, t_2 \in L(s_1, t_1)$, implying that $(s_1, t_1, s_2, t_2)$ is a rigid quadruple.

Also note that if $x \in S_1$, $u \in S_3$, $y \in S_5$, $v \in S_7$, then $(s_1, t_1, s_2, t_2) \succ (x, y, u, v)$. Indeed, we have $x \in F(s_1, s_2)$, $u \in F(s_2, t_1)$, $y \in F(t_1, t_2)$, $v \in F(t_2, s_1)$, and also $l(x, y) = 2 = l(s_1, t_1) - 2$ and $l(u, v) = 2 = l(s_2, t_2) - 2$, implying that $(x, y, u, v)$ belongs to the first expression of formula \ref{eq:Q4}.

Finally note that the number of octuples $(s_1, t_1, s_2, t_2, x, y, u, v)$ is exactly $k^8 = \Theta(|V|^8)$, so the total running time of a line \ref{line:cQ4} of the algorithm \ref{alg:n8} is at least $\Theta(|V|^8)$.
\end{proof}

Let's also estimate the space complexity of the algorithm \ref{alg:n8}
\begin{theorem}
Algorithm \ref{alg:n8} requires $\OO(|V|^4)$ of memory.
\end{theorem}
\begin{proof}
We need $\OO(|V|^2)$ of extra memory to save the shortest distance matrix and a 4-dimensional array for storing the predicate $2DSP$ values, for which we need $\OO(|V|^4)$ of memory.
\end{proof}

Thus, we can formulate the result of Eilam-Tzoreff in a following theorem: 

\begin{theorem} 
Algorithm \ref{alg:n8} calculates the predicate $2DSP(s_1, t_1, s_2, t_2)$ over all quadruples $(s_1, t_1, s_2, t_2)$ in running time of $\OO(|V|^8)$ using $\Theta(|V|^4)$ of memory. 
\end{theorem}
