\section{Model`ny'j algoritm}

Ustanovlenny'x faktov xvataet, chtoby' oformit` algoritm vy'chisleniya vsex znachenij predikata $2DSP$ v vide psevdokoda (algoritm \ref{alg:main}). 

\begin{algorithm}
\caption{Vy'chislenie vsex znachenij $2DSP(s_1, t_1, s_2, t_2)$ za $O(|V|^8)$} \label{alg:main}
\begin{algorithmic}[1]
\Procedure{CalculateAll2DSPValues}{$V$, $E$}
\State $l \gets \text{matricza poparny'x rasstoyanij mezhdu vershinami v }G$; \label{line:floyd}
\LineComment{Razob`yom pary' vershin na gruppy' v sootvetstvii s rasstoyanimi mezhdu nimi.}
\State $P_i \gets \text{pustoj spisok dlya vsex }i = 0, \ldots, |V|-1$;
\For{$s,t \in V$}
    \State $\text{dobavlyaem }(s, t)\text{ v konecz spiska }P_{l(s, t)}$;
\EndFor
\LineComment{Vy'chislim znacheniya predikata v poryadke leksikograficheskogo vozrastaniya}
\LineComment{pary' $(d_{\min}, d_{\max})$, gde $d_{\min}$ e`to men`shee iz rasstoyanij mezhdu parami terminalov,}
\LineComment{a $d_{\max}$ e`to bol`shee.}
\For{$d_{\min} \gets 0, \ldots, |V|-1$}
    \For{$d_{\max} \gets d_{\min}, \ldots, |V|-1$}
        \For{$(s_1, t_1) \in P_{d_{\min}}$}
            \For{$(s_2, t_2) \in P_{d_{\max}}$}
                \State $2DSP(s_1, t_1, s_2, t_2) \gets \textsc{CalculateSingle2DSPValue}(s_1, t_1, s_2, t_2)$;
                \State $2DSP(s_2, t_2, s_1, t_1) \gets 2DSP(s_1, t_1, s_2, t_2)$; \Comment{po simmetrii}
            \EndFor
        \EndFor
    \EndFor
\EndFor
\EndProcedure
\Statex
\Procedure{CalculateSingle2DSPValue}{$s_1$, $t_1$, $s_2$, $t_2$}
\If {$s_1 = t_1 \wedge s_2 = t_2$}
    \State \Return $s_1 = s_2$; \label{line:a}
\ElsIf {$(s_1, t_1, s_2, t_2)\text{ ne zhyostkaya chetvyorka}$}
    \State \text{vy'chislyaem i vozvrashhaem znachenie v sootvetstvii s punktom b teoremy' \ref{main_theorem}}; \label{line:b}
\Else
    \State $\text{vy'chislyaem }Q_2(s_1, t_1, s_2, t_2)\text{ po formulam \eqref{eq:Q2}}$; \label{line:cQ2}
    \State $\text{vy'chislyaem }Q_4(s_1, t_1, s_2, t_2)\text{ po formulam \eqref{eq:Q4}}$; \label{line:cQ4}
    \State \Return $Q_2(s_1, t_1, s_2, t_2) \vee Q_4(s_1, t_1, s_2, t_2)$;
\EndIf
\EndProcedure
\end{algorithmic}
\end{algorithm}

Prokommentiruem vremennuyu slozhnost` vsex sostavny'x chastej algoritma.

\begin{proposition}
Postroenie matricy' kratchajshix rasstoyanij v stroke \ref{line:floyd} algoritma \ref{alg:main} mozhno proizvesti za vremya $\OO(|V|^3)$.
\end{proposition}
\begin{proof}
Dostatochno vospol`zovat`sya algoritmom Flojda-Uorshella (\cite{Floyd}, \cite{CLRS}).
\end{proof}

\begin{proposition}
Procedura \textsc{CalculateSingle2DSPValue} rabotaet za $\OO(1)$, $\OO(|V|)$ ili $\OO(|V|^4)$ v zavisimosti ot togo, kakoj sluchaj iz teoremy' \ref{main_theorem} realizuetsya.
\end{proposition}
\begin{proof}
Legko videt`, chto stroka \ref{line:a} sostoit iz edinstvennoj instrukcii i sootvetstvuet sluchayu, trebuyushhemu konstantnoe kolichestvo operacij, stroka \ref{line:b} sootvetstvuet sluchayu, v kotorom vy'chislenie zavisit ot $\OO(|V|)$ chetvyorok (sm. punkt b teoremy' \ref{main_theorem}). 

Stroka \ref{line:cQ2} sootvetstvuet gruppe tex chetvyorok vershin, ot kotory'x zavisit chetvyorka $(s_1, t_1, s_2, t_2)$, i kotory'e otlichayutsya ot nashej chetvyorki v dvux poziciyax, stalo by't`, vy'chislenie $Q_2(s_1, t_1, s_2, t_2)$ trebuet ne bolee $\OO(|V|^2)$ operacij v xudshem sluchae.

Nakonecz, stroka \ref{line:cQ4} sootvetstvuet gruppe tex chetvyorok vershin, kotory'e otlichayutsya ot chetvyorki $(s_1, t_1, s_2, t_2)$ po kazhdoj komponente, a znachit, vy'chislenie $Q_4(s_1, t_1, s_2, t_2)$ trebuet ne bolee $\OO(|V|^4)$ operacij v xudshem sluchae.
\end{proof}

Takim obrazom, my' poluchili verxnyuyu ocenku na vremya raboty' algoritma \ref{alg:main}, kotoruyu mozhno sformulirovat` v vide teoremy':

\begin{theorem}
Vremya raboty' algoritma \ref{alg:main} est` $\OO(|V|^8)$.
\end{theorem}
\begin{proof}
Osnovnoe vremya zanimayut vy'zovy' \textsc{CalculateSingle2DSPValue}, kotory'x budet proizvedeno rovno $|V|^4$, i kazhdy'j iz kotory'x rabotaet za vremya $O(|V|^4)$.
\end{proof}

Vidno, chto samy'm uzkim mestom v programme yavlyaetsya vy'chislenie $Q_4(s_1, t_1, s_2, t_2)$. Pokazhem takzhe, chto v xudshem sluchae algoritm \ref{alg:main} rabotaet za $\Theta(|V|^8)$. Dlya e`togo dostatochno pred``yavit` semejstvo grafov skol` ugodno bol`shogo razmera, dlya kotory'x vremya raboty' algoritma \ref{alg:main} est` $\Theta(|V|^8)$.

\begin{theorem}
Postroim graf $G_k = (V_k, E_k)$, gde 

\begin{align}
V_k &= \bigsqcup\limits_{i = 0,\ldots,7} S_i \sqcup \{c\} \\
E_k &= \bigsqcup\limits_{i = 0,\ldots,7} S_i \times S_{(i+1) \bmod 8} \quad\sqcup\quad \bigsqcup\limits_{i=1,3,5,7} S_i \times \{c\} \\
|S_i| &= k.
\end{align}

\begin{figure}[H]
\caption{Graf, dlya kotorogo chast` $Q_4$ yavlyaetsya opredelyayushhej pri vy'chislenii $2DSP(s_1, t_1, s_2, t_2)$}
\centering
\includegraphics[width=0.5\textwidth]{pic.1}
\end{figure}

Zametim, chto v dannom grafe $|V_k| = 12k + 1$ i $|E_k| = 12k^2 + 4k$. Togda vremya raboty' algoritma \ref{alg:main} na grafe $G_k$ sostavit $\Theta(k^8) = \Theta(|V|^8)$.

\end{theorem}
\begin{proof}
Predpolozhim, $s_1 \in S_0$, $s_2 \in S_2$, $t_1 \in S_4$, $t_2 \in S_6$. Togda $l(s_1, t_1) = l(s_2, t_2) = 4$, $l(s_1, s_2) = l(s_2, t_1) = l(t_1, t_2) = l(t_2, s_1) = 2$, to est`, $s_1, t_1 \in L(s_2, t_2)$ i $s_2, t_2 \in L(s_1, t_1)$, a znachit, $(s_1, t_1, s_2, t_2)$ --- zhyostkaya chetvyorka vershin.

Zametim takzhe, chto esli $x \in S_1$, $u \in S_3$, $y \in S_5$, $v \in S_7$, to $(s_1, t_1, s_2, t_2) \succ (x, y, u, v)$. Dejstvitel`no, verny' sootnosheniya $x \in F(s_1, s_2)$, $u \in F(s_2, t_1)$, $y \in F(t_1, t_2)$, $v \in F(t_2, s_1)$, i pri e`tom $l(x, y) = 2 = l(s_1, t_1) - 2$ i $l(u, v) = 2 = l(s_2, t_2) - 2$, znachit, $(x, y, u, v)$ popadaet v pervoe vy'razhenie formuly' \ref{eq:Q4}.

Nakonecz zametim, chto opisanny'x naborov $(s_1, t_1, s_2, t_2, x, y, u, v)$ v tochnosti $k^8 = \Theta(|V|^8)$, a znachit, summarnoe vremya, zatrachennoe na ispolnenie stroki \ref{line:cQ4} algoritma \ref{alg:main}, sostavit $\Theta(|V|^8)$.
\end{proof}

Nakonecz, ocenim pamyat`, potreblyaemuyu algoritmom \ref{alg:main}
\begin{theorem}
Algoritm \ref{alg:main} potreblyaet $\OO(|V|^4)$ pamyati.
\end{theorem}
\begin{proof}
Dejstvitel`no, vsyo potreblenie pamyati sosredotocheno v soxranenii grafa i matricy' kratchajshix putej, na kotory'e nuzhno $\OO(|V|^2)$, pamyati, a takzhe chety'ryoxmernogo massiva logicheskix znachenij predikata $2DSP$, na kotory'j nuzhno $\OO(|V|^4)$ pamyati.
\end{proof}
