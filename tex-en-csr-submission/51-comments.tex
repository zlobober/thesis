\clearpage
\section{Appendix 1. Original algorithm by Eilam-Tzoreff}

Consider graph $G$ to be unweighted, i.e. $w(e) = 1$ for any edge $e \in E$. In such case the length of any path is equal to the number of edges in it.

Part 1 of theorem \ref{main_theorem} provides the basis for a recursion: indeed, for the degenerate case when both pairs of terminals are coinciding, the answer to the problem depends on whether these pairs define the same vertex or not.

Part 2 of theorem \ref{main_theorem} is also pretty clear: let's provide an inductive argument. For example, suppose that $s_1 \notin L(s_2, t_2)$, $s_1 \neq s_2$ and at the same time $2DSP(x, t_1, s_2, t_2) = 1$ for some $x \in F(s_1, t_1)$. Consider a pair of shortest paths $(P_1', P_2)$ that exists for pair of terminals $(x, t_1)$ and $(s_2, t_2)$. Let us show how $P_1'$ may be extended into a new path $P_1$ connecting $s_1$ and $t_1$ in a shortest manner that is disjoint from $P_2$ (which will prove that $2DSP(s_1, t_1, s_2, t_2)$ is indeed equal to $1$).

Let $P_1 = \{s_1\} \sqcup P_1'$. Note that $P_1' \cap P_2 = \varnothing$, and at the same time $s_1 \notin L(s_2, t_2) \supseteq P_2$. Hence, we have $P_1 \cap P_2 = \varnothing$. On the other hand, we have that, if $P_1, P_2$ is the sought pair of shortest paths connecting $s_1$ with $t_1$ and $s_2$ with $t_2$, we can take $x$ be equal to the successor of $s_1$ in $P_1$ and obtain that $2DSP(x, t_1, s_2, t_2) = 1$ as $P_1' = P_1 \sm \{s_1\}$ and $P_2$ is a pair of disjoint shortest paths.

Part 3 of the theorem \ref{main_theorem} describes cases when the inductive argument is not applicable. In \cite{ET} Eilam-Tzoreff performs a rigorous case analysis, which implies that $2DSP(s_1, t_1, s_2, t_2)$  may be defined by the value of $2DSP$ at 18 groups of vertex quadruples, components of which either coincide with the corresponding components of $(s_1, t_1, s_2, t_2)$ or are adjacent with them in a specific manner. Let us show some properties of the formulae in this part of theorem.

\begin{proposition} \label{q2_structure}
Each expression compounding $Q_2$ has exactly two arguments coinciding with those from the left side, and the remaining two arguments correspond to their successor sets on the shortest paths to somewhere. Moreover, for each pair of terminals $(s_i, t_i)$ exactly one terminal is left intact and the remaining one is replaced with a successor set closer to the opposing terminal vertex.
\end{proposition}

\begin{proposition}
Both expressions compounding $Q_4$ contain a successor set to the terminal as a corresponding argument for each of 4 arguments.
\end{proposition}

We will call a quadruple $(s_1, t_1, s_2, t_2)$ dependent on $(x, y, u, v)$ if a recursive formula for $2DSP(s_1, t_1, s_2, t_2)$ contains a value of $2DSP(x, y, u, v)$ in its right hand side (possibly, as a part of a complex expression in $Q_2$ or in $Q_4$). We will denote such relation as $(s_1, t_1, s_2, t_2) \succ (x, y, u, v)$.

\begin{proposition}
The relation $\succ$ is acyclic. In other words, using \ref{main_theorem} we can calculate all values of $2DSP$ at all points using the dynamic programming approach.
\end{proposition}
\begin{proof}
In the part 1 of theorem \ref{main_theorem} the right-hand side for $2DSP(s_1, t_1, s_2, t_2)$ does not contain any other values or predicate, so we can conclude that such quadruples $(s_1, t_1, s_2, t_2)$ do not depend on anything.

For the parts 2 and 3 note the following property: if $(s_1, t_1, s_2, t_2) \succ (x, y, u, v)$, then $l(x, y) \leq l(s_1, t_1)$, $l(u, v) \leq l(s_2, t_2)$ and at least one of this inequalities is strict. For $Q_2$ this is correct due to proposotion \ref{q2_structure} about structure of expressions compounding $Q_2$, and for $Q_4$ this is correct due to fact that for valid quadruple $(x, y, u, v)$ $l(x, y) = l(s_1, t_1) - 2$ and $l(u, v) = l(s_2, t_2) - 2$.
\end{proof}

Finally, let's prove an important property of rigid quadruples.

\begin{proposition} \label{eql_ap}
For a rigid quadruple $(s_1, t_1, s_2, t_2)$ it is true that $l(s_1, t_1) = l(s_2, t_2)$.
\end{proposition}
\begin{proof} 

$s_1, t_1 \in L(s_2, t_2) \Rightarrow l(s_1, s_2) + l(s_2, t_1) = l(s_1, t_2) + l(t_2, t_1)$, since both expressions are equal to $l(s_1, t_1)$.

$s_2, t_2 \in L(s_1, t_1) \Rightarrow l(s_2, s_1) + l(s_1, t_2) = l(s_2, t_1) + l(t_1, t_2)$, since both expressions are equal to $l(s_2, t_2)$.

By adding together two last equations and canceling the same terms, we obtain the equation $l(s_1, s_2) = l(t_2, t_1)$. 

By replacing $l(s_1, s_2)$ with $l(t_2, t_1)$ in the first equation, we also obtain that $l(s_2, t_1) = l(s_1, t_2)$.

Finally, note that $l(s_1, t_1) = l(s_1, s_2) + l(s_2, t_1) = l(t_1, t_2) + l(s_2, t_1) = l(s_2, t_2)$, which finishes the proof.
\end{proof}

