\section{Appendix 3. Running time and space complexity analysis}

\subsection{$\OO(|V|^7)$ algorithm}

\begin{proposition} \label{n7_corr1}
By the moment of calling $\textsc{CalculateAllExYVValues}(d)$, all needed values of $2DSP$ are already calculated.
\end{proposition}
\begin{proof}
Note that $\textsc{CalculateAllExYVValues}(d)$ operates only with values of $2DSP(x, y, u, v)$ for $(x, y, u, v)$ s.t. $x \in F(s_1, t_1)$ and $y \in F(t_1,x,t_2) \subseteq F(t_1, x)$, i.e. $l(x, y) = l(s_1, t_1) - 2 = d - 2$, and hence, $\min\{l(x, y), l(u, v)\} \leq l(x, y) = d - 2 < d$. Also note that the only call to $\textsc{CalculateAllExYVValues}(d)$ is located at line \ref{line:calcExYV} and the argument $d$ is equal to $d_{\min}$. This means that the values of $2DSP(x, y, u, v)$ for quadruples $(x, y, u, v)$ s.t. $\min\{l(x, y), l(u, v)\} < d$ are already calculated. 
\end{proof}

\begin{proposition} \label{n7_corr2}
By the moment line \ref{line:useExYV} of Algorithm \ref{alg:n7} is run, all needed values of $ExYV$ are already calculated.
\end{proposition}
\begin{proof}
Due to Lemma \ref{eql}, we know that $l(s_1, t_1)$, $l(s_2, t_2)$ are equal, so they should both be equal to $d_{\min}$. The last call to $\textsc{CalculateAllExYVValues}$ was for $d = d_{\min}$, so all values of $ExYV(\cdot, \cdot, s, t, \cdot)$ for all $(s, t)$ s.t. $l(s, t) \leq d_{\min}$ are already calculated. Finally note the fact that the third and the fourth arguments of a predicate $ExYV$ in line \ref{line:useExYV} may only be $(s_1, t_1)$ and $(s_2, t_2)$.
\end{proof}

Let us estimate the complexity of the obtained algorithm.

\begin{proposition} \label{n7_rt1}
The total running time of $\textsc{CalculateAllExYVValues}$ for all $d = 0, \ldots, |V|-1$ is $\OO(|V|^7)$. 
\end{proposition}
\begin{proof}
Note that each pair $(s_1, t_1)$ will be processed only when $d = l(s_1, t_1)$, so the total running time of this procedure does not exceed the number of septuples $(s_1, t_1, x, u, t_2, y, v)$, i.e. $|V|^7$.
\end{proof}

\begin{proposition} \label{n7_rt2}
Each call to $\textsc{CalculateSingle2DSPValue}$ takes $\OO(|V|^2)$ time.
\end{proposition}
\begin{proof}
This is obvious from the formula used to process the rigid quadruples in line \ref{line:useExYV}.
\end{proof}

Let us also estimate the space complexity.

\begin{proposition} \label{n7_m}
Algorithm \ref{alg:n7} uses $\OO(|V|^5)$ memory.
\end{proposition}
\begin{proof}
In addition to the memory used by the original algorithm, we introduce a new 5-dimensional binary array $ExYV$, which requires $\Theta(|V|^5)$ memory.
\end{proof}

So, the final result of this section is the following
\begin{theorem}
Algorithm \ref{alg:n7} calculates the predicate $2DSP(s_1, t_1, s_2, t_2)$ for all quadruples $(s_1, t_1, s_2, t_2)$ in time $\OO(|V|^7)$ using $\Theta(|V|^5)$ memory. 
\end{theorem}

\subsection{$\OO(|V|^6)$ algorithm}

\begin{proposition}
The total running time of $\textsc{CalculateAllExYVValues}$ for all $d = 0, \ldots, |V|-1$ is $\OO(|V|^6)$. 
\end{proposition}
\begin{proof}
In the first outer loop, each pair $(u, t_2)$ is processed for exactly one value of $d = l(u, t_2) + 1$, so the total running time of the first loop does not exceed the number of quadruples $(u, t_2, x, s_1)$ multiplied by the running time of $\OO(|V|^2)$ of line \ref{line:psi} yielding the upper bound of $\OO(|V|^6)$.

In the second outer loop, each pair $(s_1, t_1)$ is processed for exactly one value of $d = l(s_1, t_1)$, so the total running time of the second loop does not exceed the number of quintuples $(s_1, t_1, x, u, t_2)$ multiplied by the running time $\OO(|V|)$ of line \ref{line:ExYVpsi} yielding the same upper bound of $\OO(|V|^6)$.
\end{proof}

\begin{proposition}
The total memory usage of Algorithm \ref{alg:n6} is $\OO(|V|^5)$.
\end{proposition}
\begin{proof}
In addition to the memory required for \ref{alg:n7}, we also need a 4-dimensional array of vectors $\psi$, which requires $\OO(|V|^5)$ memory.
\end{proof}

Hence the final result of this section is the following theorem:
\begin{theorem}
Algorithm \ref{alg:n6} calculates predicate $2DSP(s_1, t_1, s_2, t_2)$ for all quadruples $(s_1, t_1, s_2, t_2)$ in time $O(|V|^6)$ using $\Theta(|V|^5)$ memory. 
\end{theorem}
