\section{Reducing running time to $\OO(|V|^7)$}

In the following two sections we assume edges to have unit length, i.e. the graph to be unweighted.

We start by optimizing the DP transitions given by \eqref{eq:Q4} as all the remaining transitions require $\OO(|V|^6)$ time. Denote the first and the second expression in \eqref{eq:Q4} by $A_4(s_1, t_1, s_2, t_2)$ and $B_4(s_1, t_1, s_2, t_2)$:
\begin{align} \label{eq:A}
    A(s_1, t_1, s_2, t_2) = \quad
        \smashoperator{\bigvee_{
        \large \substack{
            x \in F(s_1, s_2)\\ 
            y \in F(t_1, t_2)\\ 
            u \in F(s_2, t_1)\\ 
            v \in F(t_2, s_1)\\ 
            l(x, y) + 2 = l(s_1, t_1)\\ 
            l(u, v) + 2 = l(s_2, t_2)
        }
        }} 2DSP(x, y, u, v) \\
\end{align}
\begin{align} \label{eq:B}
   B(s_1, t_1, s_2, t_2) = \quad
        \smashoperator{\bigvee_{
        \large \substack{
            x \in F(s_1, t_2)\\ 
            y \in F(t_1, s_2)\\ 
            u \in F(s_2, s_1)\\ 
            v \in F(t_2, t_1)\\ 
            l(x, y) + 2 = l(s_1, t_1)\\ 
            l(u, v) + 2 = l(s_2, t_2)
        }
        }} 2DSP(x, y, u, v) \\
\end{align}
\begin{proposition}
The following equation holds:
\begin{equation}
    B(s_1, t_1, s_2, t_2) = A(s_2, t_2, s_1, t_1).
\end{equation}
\end{proposition}
\begin{proof}
The equation immediately follows from the existence of a bijection between quadruples $(x, y, u, v)$ s.t. $x \in F(s_2, s_1)$, $y \in F(t_2, t_1)$, $u \in F(s_1, t_2)$, $v \in F(t_1, s_2)$, $l(x, y) + 2 = l(s_2, t_2)$, $l(u, v) + 2 = l(s_1, t_1)$ (that form the disjunction domain in the expression for $A(s_2, t_2, s_1, t_1)$) and quadruples $(x', y', u', v')$ s.t. $x' \in F(s_1, t_2)$, $y' \in F(t_1, s_2)$, $u' \in F(s_2, s_1)$, $v' \in F(t_2, t_1)$, $l(x', y') + 2 = l(s_1, t_1)$, $l(u', v') + 2 = l(s_2, t_2)$. The sought bijection is defined by $x' = u$, $y' = v$, $u' = x$, $v' = y$.
\end{proof}

Thus, $Q_4(s_1, t_1, s_2, t_2)$ may be expressed using just predicate $A$:

\begin{equation} \label{eq:Q4A}
Q_4(s_1, t_1, s_2, t_2) = A(s_1, t_1, s_2, t_2) \vee A(s_2, t_2, s_1, t_1)
\end{equation}

We rewrite the conditions on $(x, y, u, v)$ in \eqref{eq:A} in the equivalent form:

\begin{proposition}

\begin{equation}
\begin{cases}
    x \in F(s_1, s_2)\\ 
    y \in F(t_1, t_2)\\ 
    u \in F(s_2, t_1)\\ 
    v \in F(t_2, s_1)\\ 
    l(x, y) + 2 = l(s_1, t_1)\\ 
    l(u, v) + 2 = l(s_2, t_2)
\end{cases} \Longleftrightarrow
\begin{cases}
    x \in F(s_1, t_1, s_2)\\ 
    y \in F(t_1, s_1, t_2)\\ 
    u \in F(s_2, t_2, t_1)\\ 
    v \in F(t_2, s_2, s_1)\\ 
    l(x, y) + 2 = l(s_1, t_1)\\ 
    l(u, v) + 2 = l(s_2, t_2)
\end{cases} \Longleftrightarrow
\begin{cases}
    x \in F(s_1, t_1, s_2)\\ 
    y \in F(t_1, x, t_2)\\ 
    u \in F(s_2, t_2, t_1)\\ 
    v \in F(t_2, u, s_1)\\ 
\end{cases}
\end{equation}
\end{proposition}
\begin{proof}

First, note that $l(x, y) + 2 = l(s_1, t_1)$ and $l(u, v) + 2 = l(s_2, t_2)$ immediately implies that $x, y \in L(s_1, t_1)$ and $u, v \in L(s_2, t_2)$, so we may replace $F(s_1, s_2)$ with $F(s_1, t_1, s_2)$ as a domain for $x$, and perform the similar action for $y$, $u$ and $v$. This proves the first equivalence.

The second equivalence may be proven by exploiting the fact that $l(x, y) + 2 = l(s_1, t_1)$ (which means that $x$ and $y$ belong to the same shortest path between $s_1$ and $t_1$) is equivalent to $x \in F(s_1, t_1) \wedge y \in F(t_1, x)$, and similarly for $s_2$, $t_2$, $u$ and $v$. 
\end{proof}

Now we can rewrite \eqref{eq:A} as follows:
\begin{equation} 
    A(s_1, t_1, s_2, t_2) = \quad
        {\bigvee_{
        \large \substack{
            x \in F(s_1, t_1, s_2)\\ 
            u \in F(s_2, t_2, t_1) 
        }
        }}
        \quad
        {\bigvee_{
        \large \substack{
            y \in F(t_1, x, t_2)\\ 
            v \in F(t_2, u, s_1) 
        }
        }} 2DSP(x, y, u, v)
\end{equation}

Note the important property of the equation above: the expression inside the first disjunction does not depend on $s_2$. Let us denote it by $ExYV(x, u, s_1, t_1, t_2)$ (an acronym for ``there exist $y$ and $v$''):

\begin{align} \label{eq:ExYV}
    A(s_1, t_1, s_2, t_2) &= \quad
       \smashoperator{\bigvee_{
        \large \substack{
            x \in F(s_1, t_1, s_2)\\ 
            u \in F(s_2, t_2, t_1) 
        }
        }} ExYV(x, u, s_1, t_1, t_2) = ExYV(F(s_1, t_1, s_2), F(s_2, t_2, t_1), s_1, t_1, t_2)\\
    ExYV(x, u, s_1, t_1, t_2) &= \quad
        \smashoperator{\bigvee_{
        \large \substack{
            y \in F(t_1, x, t_2)\\ 
            v \in F(t_2, u, s_1) 
        }
        }} 2DSP(x, y, u, v) = 2DSP(x, F(t_1, x, t_2), u, F(t_2, u, s_1))\\ 
\end{align}

Formulae above suggest the following optimization: $A(s_1, t_1, s_2, t_2)$ only depends on $\OO(|V|^2)$ values of $ExYV(x, u, s_1, t_1, t_2)$. So, if we have all the necessary values of $ExYV(x, u, s_1, t_1, t_2)$ ready by the moment we calculate the value of $A(s_1, t_1, s_2, t_2)$, then the total running time needed for calculating all $A(s_1, t_1, s_2, t_2)$ becomes $\OO(|V|^6)$.

Finally, note that by \eqref{eq:ExYV} we may calculate each value of $ExYV(x, u, s_1, t_1, t_2)$ in $\OO(|V|^2)$ time. This means that $ExYV(x, u, s_1, t_1, t_2)$ is a suitable auxilary predicate that, on one hand, may be calculated efficiently (in $\OO(|V|^7)$ time), and on the other hand, is reused multiple times while calculating values of predicate $A(s_1, t_1, s_2, t_2)$, providing the sought running time optimization.

The only remaining step is to verify that we can organize the calculations in an appropriate order; i.e. by the moment we use each of the values in the right-hand sides of \eqref{eq:Q4A} and \eqref{eq:ExYV}, it is already known. To show this, we provide a pseudocode of Algorithm \ref{alg:n7} implementing this approach.

\begin{algorithm}
\caption{Calculation of $2DSP(s_1, t_1, s_2, t_2)$ in $\OO(|V|^7)$} \label{alg:n7}
\begin{algorithmic}[1]
\Procedure{CalculateAll2DSPValues}{$V$, $E$}
\State $l \gets \text{matrix of pairwise distances in }G$
\LineComment{Group pairs of vertices according to the distance between them.}
\State $P_i \gets \text{empty list for all }i = 0, \ldots, |V|-1$
\For{$s,t \in V$}
    \State $\text{Append }(s, t)\text{ to }P_{l(s, t)}$
\EndFor
\LineComment{Calculate values of the predicate in lexicographical order of pairs }
\LineComment{$(d_{\min}, d_{\max})$, where $d_{\min}$ is for the smallest of the distances}
\LineComment{between terminal pairs and $d_{\max}$ is for the largest of them.}
\For{$d_{\min} \gets 0, \ldots, |V|-1$}
    \State $\textsc{CalculateAllExYVValues}(d_{\min})$ \label{line:calcExYV}
    \For{$d_{\max} \gets d_{\min}, \ldots, |V|-1$}
        \For{$(s_1, t_1) \in P_{d_{\min}}$}
            \For{$(s_2, t_2) \in P_{d_{\max}}$}
                \State $2DSP(s_1, t_1, s_2, t_2) \gets \textsc{CalculateSingle2DSPValue}(s_1, t_1, s_2, t_2)$
                \LineComment{Finally use the symmetric nature of our predicate.}
                \State $2DSP(s_2, t_2, s_1, t_1) \gets 2DSP(s_1, t_1, s_2, t_2)$
            \EndFor
        \EndFor
    \EndFor
\EndFor
\EndProcedure
\Statex
\Procedure{CalculateAllExYVValues}{$d$}
    \For{$(s_1,t_1 \in P_d)$}
        \For{$x \in F(s_1, t_1)$}
            \For{$u,t_2 \in V$}
                \State $ExYV(x,u,s_1,t_1,t_2) \gets 2DSP(x, F(t_1,x,t_2), u, F(t_2,u,s_1))$
            \EndFor
        \EndFor
    \EndFor
\EndProcedure
\Statex
\Procedure{CalculateSingle2DSPValue}{$s_1$, $t_1$, $s_2$, $t_2$}
\If {$s_1 = t_1 \wedge s_2 = t_2$}
    \State \Return $s_1 = s_2$
\ElsIf {$(s_1, t_1, s_2, t_2)\text{ is not rigid}$}
    \State \text{Calculate and return the value according to case (2) of Th~\ref{main_theorem}}
\Else
    \State $\text{Calculate }Q_2(s_1, t_1, s_2, t_2)\text{ using \eqref{eq:Q2}}$
    \State $Q_4(s_1, t_1, s_2, t_2) = ExYV(F(s_1, t_1, s_2), F(s_2, t_2, t_1), s_1, t_1, t_2)\enspace\vee\enspace$ 
    \Statex $\hspace{3.65cm}ExYV(F(s_2, t_2, s_1), F(s_1, t_1, t_2), s_2, t_2, t_1)$ \label{line:useExYV}
    \State \Return $Q_2(s_1, t_1, s_2, t_2) \vee Q_4(s_1, t_1, s_2, t_2)$
\EndIf
\EndProcedure
\end{algorithmic}
\end{algorithm}

\begin{proposition} \label{n7_corr1}
By the moment of calling $\textsc{CalculateAllExYVValues}(d)$, all needed values of $2DSP$ are already calculated.
\end{proposition}
\begin{proof}
Note that $\textsc{CalculateAllExYVValues}(d)$ operates only with values of $2DSP(x, y, u, v)$ for $(x, y, u, v)$ s.t. $x \in F(s_1, t_1)$ and $y \in F(t_1,x,t_2) \subseteq F(t_1, x)$, i.e. $l(x, y) = l(s_1, t_1) - 2 = d - 2$, and hence, $\min\{l(x, y), l(u, v)\} \leq l(x, y) = d - 2 < d$. Also note that the only call to $\textsc{CalculateAllExYVValues}(d)$ is located at line \ref{line:calcExYV} and the argument $d$ is equal to $d_{\min}$. This means that the values of $2DSP(x, y, u, v)$ for quadruples $(x, y, u, v)$ s.t. $\min\{l(x, y), l(u, v)\} < d$ are already calculated. 
\end{proof}

\begin{proposition} \label{n7_corr2}
By the moment line \ref{line:useExYV} of Algorithm \ref{alg:n7} is run, all needed values of $ExYV$ are already calculated.
\end{proposition}
\begin{proof}
Due to Lemma \ref{eql}, we know that $l(s_1, t_1)$, $l(s_2, t_2)$ are equal, so they should both be equal to $d_{\min}$. The last call to $\textsc{CalculateAllExYVValues}$ was for $d = d_{\min}$, so all values of $ExYV(\cdot, \cdot, s, t, \cdot)$ for all $(s, t)$ s.t. $l(s, t) \leq d_{\min}$ are already calculated. Finally note the fact that the third and the fourth arguments of a predicate $ExYV$ in line \ref{line:useExYV} may only be $(s_1, t_1)$ and $(s_2, t_2)$.
\end{proof}

Let us estimate the complexity of the obtained algorithm.

\begin{proposition} \label{n7_rt1}
The total running time of $\textsc{CalculateAllExYVValues}$ for all $d = 0, \ldots, |V|-1$ is $\OO(|V|^7)$. 
\end{proposition}
\begin{proof}
Note that each pair $(s_1, t_1)$ will be processed only when $d = l(s_1, t_1)$, so the total running time of this procedure does not exceed the number of septuples $(s_1, t_1, x, u, t_2, y, v)$, i.e. $|V|^7$.
\end{proof}

\begin{proposition} \label{n7_rt2}
Each call to $\textsc{CalculateSingle2DSPValue}$ takes $\OO(|V|^2)$ time.
\end{proposition}
\begin{proof}
This is obvious from the formula used to process the rigid quadruples in line \ref{line:useExYV}.
\end{proof}

Let us also estimate the space complexity.

\begin{proposition} \label{n7_m}
Algorithm \ref{alg:n7} uses $\OO(|V|^5)$ memory.
\end{proposition}
\begin{proof}
In addition to the memory used by the original algorithm, we introduce a new 5-dimensional binary array $ExYV$, which requires $\Theta(|V|^5)$ memory.
\end{proof}

So, the final result of this section is the following
\begin{theorem}
Algorithm \ref{alg:n7} calculates the predicate $2DSP(s_1, t_1, s_2, t_2)$ for all quadruples $(s_1, t_1, s_2, t_2)$ in time $\OO(|V|^7)$ using $\Theta(|V|^5)$ memory. 
\end{theorem}
