\section{Introduction}

Consider the following general combinatorial problem: \emph{Given a graph $G = (V, E)$ and $k$ pairs of vertices $(s_i, t_i)$ ($s_i, t_i \in V$, $1 \leq i \leq k$), find out if there exist a $k$-tuple of paths $P_i$ such that: 1) path $P_i$ goes from $s_i$ to $t_i$ 2) each $P_i$ is one of the shortest paths from $s_i$ to $t_i$ and 3) paths are pairwise disjoint.}

Graph $G$ above may either be directed or undirected; disjointness property may require paths to be either node-disjoint or edge-disjoint; finally the graph may be either weighted (with positive lengths assigned to edges) or unweighted (having all-unit edge lengths). Clearly there are 8 problem versions described above. For example, we may consider the undirected weighted vertex-disjoint version of $k$DSP.

In \cite{ET} it is proven that if $k$ is a part input then $k$DSP immediately becomes NP-complete even in the planar unit-length case irrespective of whether the graph is directed and whether we choose vertex-disjoint paths or edge-disjoint paths. 

In this article we will concentrate on the case $k=2$. For weighted undirected vertex-disjoint case, Eilam-Tzoreff provided a polynomial time algorithm based on a dynamic programming approach. He also provided a linear-time reduction from the edge-disjoint case to the vertex-disjoint case. The primary drawback of the algorithm of Eilam-Tzoreff is its unsatisfactory running time of $\OO(|V|^8)$. This motivates a natural question --- is it possible to solve the problem faster?

In this article we obtain the algorithm with running time of $\OO(|V|^6)$ for the unit-length case of $2$DSP and the algorithm with running time of $\OO(|V|^7)$ for the weighted case of $2$DSP (in both cases we consider the vertex-disjoint undirected formulation). Our algorithms may be viewed as modificiations of Eilam-Tzoreff algorithm with two improvements. The first one is somewhat standard to dynamic programming and consists of choosing the appropriate computation order enabling us to factor the problem into two independent subproblems with better running time. The second one is novel and works as follows: we interpret the computationally hardest subroutine of the algorithm as taking the value $x^T\beta y$ of a certain bilinear form $\beta$ at some pair of vectors $x$, $y$, and then carefully analyze the triples $\beta, x, y$ arising during the computation. It turns out that by pre-evaulating the partial products $x^T \beta$ we may reduce the running time complexity even further.

As a proof of concept, the obtained algorithms were implemented in C++. With their use, the correctness of the algorithm was checked for all possible unweighted graphs of small size ($|V| \leq 8$) and for a significant number of connected graphs of larger size ($|V| = 10, 20, 30$). 

% This paper is structured in the following manner:

% Glava 1 soderzhit vse neobxodimy'e opredeleniya iz stat`i \cite{ET}, a takzhe osnovnoj rezul`tat stat`i, na kotorom osnovy'vaetsya b\'{o}l`shaya chast` dannoj raboty'. 

% Glava 2 soderzhit ryad vspomogatel`ny'x utverzhdenij, neobxodimy'x dlya postroeniya model`nogo algoritma dlya nevzveshennogo sluchaya, s kotory'm i budet vestis` dal`nejshaya rabota po optimizacii.

% Glava 3 soderzhit model`ny'j algoritm dlya nevzveshennogo sluchaya s dokazatel`stvom nizhnej i verxnej ocenki na vremya raboty' algoritma.

% Glava 4 soderzhit nablyudeniya, neobxodimy'e, dlya umen`sheniya vremeni raboty' algoritma dlya nevzveshennj zadachi v xudshem sluchae do $\OO(|V|^7)$.

% Glava 5 soderzhit nablyudenie, neobxodimy'e, dlya umen`sheniya vremeni raboty' algoritma dlya nevzveshennoj zadachi v xudshem sluchae do $\OO(|V|^6)$.

% Glava 6 soderzhit prakticheskie rezul`taty', poluchenny'e vo vremya realizacii model`nogo algoritma dlya nevzveshennoj zadachi i ego optimizirovanny'x versij v vide programmy' na yazy'ke programmirovaniya C++.

% Glava 7 soderzhit ryad dopolnitel`ny'x nablyudenij, obobshhayushhix poluchenny'j algoritm na sluchaj vzveshenny'x grafov cenoj dopolnitel`nogo mnozhitelya v $|V|$ vo vremeni raboty', a takzhe potencial`no pomogayushhix poluchit` eshhyo bolee e`ffektivnoe reshenie. 

The source code of our implementations and the LaTeX sources of this article (both in English and in Russian) are available at \url{https://github.com/zlobober/thesis}.
