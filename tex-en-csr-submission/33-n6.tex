\section{Reducing running time to $\OO(|V|^6)$}

It is easy to see that the most inefficient part of Algorithm \ref{alg:n7} is procedure $\textsc{CalculateAllExYVValues}$ whose total running time is $\OO(|V|^7)$. In this section we optimize the running time of calculating $ExYV$ for all arguments to $\OO(|V|^6)$.

Let us rephrase \eqref{eq:ExYV} by rewriting it in terms of bilinear forms. In this section we fix some basis and identify bilinear forms their matrices; also vectors are assumed to be written as columns.

\begin{definition}
Let $V = \{v_0, v_1, \ldots, v_{n-1}\}$ be a certain fixed numbering of vertices and $S \subseteq V$. Then the \emph{characteristic vector $\chi[S]$} of set $S$ is the column vector with $1$ at the $i$-th place if $v_i \in S$ and $0$ otherwise.
\end{definition}

We claim \eqref{eq:ExYV} amounts to computing the value of a certain bilinear form at vectors $\chi[F(t_1, x, t_2)]$ and $\chi[F(t_2, u, s_1)]$ and checking if the result is zero. A large caveat here is not to rely on any of yet-to-be-calculated values while constructing the matrix of this bilinear form. 

\begin{definition}
Suppose the values of $2DSP(x, y, u, v)$ for all $(x, y, u, v)$ s.t. $\min\{l(x, y), l(u, v)\} < d$ are already calculated. Define matrix $2DSP(d,x,u) = (2DSP(d,x,u)_{i,j})_{i,j=0,\ldots,n-1}$ as follows

\begin{equation}
2DSP(d,x,u)_{i,j} = \begin{cases}
    2DSP(x,v_i,u,v_j) & \text{if } l(x,v_i) < d \text{ and } l(u,v_j) < d \\
    0                 & \text{otherwise}
\end{cases}
\end{equation}

\end{definition}

Now we can rewrite \eqref{eq:ExYV}.

\begin{proposition}
Suppose the values of $2DSP(x, y, u, v)$ for all $(x, y, u, v)$ s.t. $\min\{l(x, y), l(u, v)\} < d$ are already calculated. Suppose that $l(x, t_1) = l(u, t_2) = d - 1$. Then the following equation holds:
\begin{equation}
\label{eq:ExYVbil}
ExYV(x, u, s_1, t_1, t_2) = \begin{cases}
     1 & \text{if } \chi[F(t_1, x, t_2)]^T~2DSP(d,x,u)~\chi[F(t_2, u, s_1)] > 0 \\
     0 & \text{otherwise}
\end{cases}
\end{equation}
\end{proposition}
\begin{proof}
Suppose there exist $y$ and $v$ producing a positive value of \eqref{eq:ExYV}, namely $y = v_i$ and $v = v_j$. Note that $\chi[F(t_1, x, t_2)]_i = \chi[F(t_2, u, s_1)]_j = 1$ as $y \in F(t_1, x, t_2)$ and $v \in F(t_2, u, s_1)$ due to variable domain in \eqref{eq:ExYV}. Also note that $2DSP(d, x, u)_{i,j} = 2DSP(x, y, u, v) = 1$ (as $l(x, y) = l(u, v) = d - 2 < d$). Hence, the value of the bilinear form at our pair of characteristic vectors is at least 1.

By using exactly the same argument in the reverse direction we show that the right-hand side being at least 1 implies that there exist the desired $y$ and $v$, finishing the proof. 
\end{proof}

We did not achieve any significant speedup yet as we have only rewritten the same formulae in a different way. The key idea is to carefully estimate the number of pairs of bilinear forms and its its right-hand vector argument. 

\begin{proposition}
There are $\OO(|V|^3)$ bilinear forms that we consider during our algorithm.
\end{proposition}
\begin{proof}
Immediately follows from the fact that each form corresponds to a triple of $(d, x, u)$, each of whose component takes $|V|$ possible values.
\end{proof}

\begin{proposition}
There are $\OO(|V|^4)$ pairs of bilinear forms and its right-hand vector arguments that we consider during our algorithm.
\end{proposition}
\begin{proof}
A pair of the above type is $(2DSP(d, x, u), \chi[F(t_2, u, s_1)])$ with the extra condition of $l(u, t_2) = d - 1$. So, the desired number of pairs is bounded by the number of quintuples $(x, u, t_2, s_1)$, which is $|V|^4$.
\end{proof}

Define the value of $\psi(x, u, t_2, s_1)$ as follows:

\begin{equation}
\psi(x, u, t_2, s_1) = 2DSP(l(u, t_2) + 1, x, u)~\chi[F(t_2, u, s_1)] \label{eq:psi}
\end{equation}

The formula above provides an upper time bound of $\OO(|V|^6)$ for calculating all $\psi(x, u, t_2, s_1)$, as each single value may be calculated in $\OO(|V|^2)$ by multiplying a matrix by a vector in a straightforward manner. 

Formula \eqref{eq:ExYVbil} takes the following form:

\begin{equation}
ExYV(x, u, s_1, t_1, t_2) = \chi[F(t_1, x, t_2)]^T \psi(x, u, t_2, s_1) \label{eq:ExYVpsi}
\end{equation}

Using the new formula, we may calculate $ExYV(x, u, s_1, t_1, t_2)$ in time $\OO(|V|)$, so the total time of calculating all $ExYV(x, u, s_1, t_1, t_2)$ becomes $\OO(|V|^6)$.

As in the previous section, we provide the pseudocode for our approach (Algoritm \ref{alg:n6}) and analyze its running time and space complexity.

\begin{algorithm}
\caption{Calculation of all $ExYV(x, u, s_1, t_1, t_2)$ in $\OO(|V|^6)$} \label{alg:n6}
\begin{algorithmic}[1]
\Procedure{CalculateAllExYVValues}{$d$}
    \For{$(u, t_2) \in P_{d-1}$}
        \For{$x, s_1 \in V$}
            \State $\psi(x, u, t_2, s_1) \gets 2DSP(l(u, t_2) + 1, x, u)~\chi[F(t_2, u, s_1)] \label{line:psi}$
        \EndFor
    \EndFor
    \For{$(s_1,t_1 \in P_d)$}
        \For{$x \in F(s_1, t_1)$}
            \For{$u,t_2 \in V$}
                \State $ExYV(x, u, s_1, t_1, t_2) \gets \chi[F(t_1, x, t_2)]^T \psi(x, u, t_2, s_1)$ \label{line:ExYVpsi}
            \EndFor
        \EndFor
    \EndFor
\EndProcedure
\end{algorithmic}
\end{algorithm}

\begin{proposition}
The total running time of $\textsc{CalculateAllExYVValues}$ for all $d = 0, \ldots, |V|-1$ is $\OO(|V|^6)$. 
\end{proposition}
\begin{proof}
In the first outer loop, each pair $(u, t_2)$ is processed for exactly one value of $d = l(u, t_2) + 1$, so the total running time of the first loop does not exceed the number of quadruples $(u, t_2, x, s_1)$ multiplied by the running time of $\OO(|V|^2)$ of line \ref{line:psi} yielding the upper bound of $\OO(|V|^6)$.

In the second outer loop, each pair $(s_1, t_1)$ is processed for exactly one value of $d = l(s_1, t_1)$, so the total running time of the second loop does not exceed the number of quintuples $(s_1, t_1, x, u, t_2)$ multiplied by the running time $\OO(|V|)$ of line \ref{line:ExYVpsi} yielding the same upper bound of $\OO(|V|^6)$.
\end{proof}

\begin{proposition}
The total memory usage of Algorithm \ref{alg:n6} is $\OO(|V|^5)$.
\end{proposition}
\begin{proof}
In addition to the memory required for \ref{alg:n7}, we also need a 4-dimensional array of vectors $\psi$, which requires $\OO(|V|^5)$ memory.
\end{proof}

Hence the final result of this section is the following theorem:
\begin{theorem}
Algorithm \ref{alg:n6} calculates predicate $2DSP(s_1, t_1, s_2, t_2)$ forr all quadruples $(s_1, t_1, s_2, t_2)$ in time $O(|V|^6)$ using $\Theta(|V|^5)$ memory. 
\end{theorem}

Let us also note an interesting feature of the obtained algorithm. In line \ref{line:psi} we repeatedly multiply the same matrix by a large number of column vectors. For each matrix, if we group $\OO(|V|)$ possible vectors into a separate matrix, the desired procedure becomes expressible in terms of matrix multiplication, which may be done more efficiently in time $\OO(|V|^\omega)$ with $\omega < 2.3727$ (\cite{Williams}). But even if we use this observation and reduce the running time of this part of the algorithm to $\OO(|V|^{{3 + \omega}})$, computing the values of $Q_2$ (which takes $\OO(|V|^6)$ time in total) becomes the bottleneck.
