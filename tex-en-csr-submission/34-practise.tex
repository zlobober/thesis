\section{Experimental evaluation}

We implemented 4 algorithms:

\begin{itemize}
\item Original algorithm \ref{alg:n8} working in $\OO(|V|^8)$;
\item Algorithm \ref{alg:n7} working in $\OO(|V|^7)$;
\item Algorithm \ref{alg:n6} working in $\OO(|V|^6)$;
\item Bruteforce algorithm \ref{alg:brute} working in exponential running time $\OO(2^{|V|} \cdot |V|)$, not relying on the theorem \ref{main_theorem}.
\end{itemize}

\begin{algorithm}
\caption{Calculation of all $2DSP(s_1, t_1, s_2, t_2)$ in $O(2^{|V|} \cdot |V|)$} \label{alg:brute}
\begin{algorithmic}[1]
\Procedure{CalculateAll2DSPValues}{$V$, $E$}
\State $l \gets \text{matrix of pairwise distances in }G$;
\For{$s_1, t_1, s_2, t_2, \in V$}
    \State $2DSP(s_1, t_1, s_2, t_2) = \textsc{CalculateSingle2DSPValue}(s_1, t_1, s_2, t_2)$;
\EndFor
\EndProcedure
\Statex
\Procedure{CalculateSingle2DSPValue}{$s_1$, $t_1$, $s_2$, $t_2$}
\State $2DSP(s_1, t_1, s_2, t_2) \gets 0$;
\For{\text{shortest path }$P\text{ between }s_1\text{ and }t_1$}
    \If {$\text{distance between }s_2\text{ and }t_2\text{ in }G \sm P = l(s_2, t_2)$}
        \State $2DSP(s_1, t_1, s_2, t_2) \gets 1$;
    \EndIf
\EndFor
\EndProcedure
\end{algorithmic}
\end{algorithm}

All mentioned algorithms were implemented as the routines in a single program, allowing their simulateneous evaluation on the same graph $G$ (either provided or randomly generated from some probability distribution). The program was run on all enumerated connected graphs consisting of no more than $8$ vertices that helped to find the enormous number of mistakes in the implementation of formula \eqref{eq:Q2}. The number of enumerated connected graphs on $9$ vertices exceeds $6 \cdot 10^{10}$, so the experimental evaluation of the algorithms on all $9$ vertex graphs with only one execution thread is not practically possible. 

As a next step, we ran $10^4$ instances of a program at the computational cluster of Yandex company, each of which evaluated $10^5$ random $10$-vertex graphs and assered that all implemented algorithms  produce the same result. Such stress test discovered a few more mistakes in the implementation of the original algorithm \ref{alg:n8} and its optimized versions. Interesting detail was that all of the mistakes were located exactly in the hardest case of rigid quadruple $(s_1, t_1, s_2, t_2)$ (formula \eqref{eq:Q4}). Hence, we can conclude that the minimum size of the graph, that triggers the hardest case of the approach of Eilam-Tzoreff is either $9$ or $10$. An example of a ``complex'' graph $H$ consisting of $10$ vertices, for which the transitions defined by the formula \eqref{eq:Q4} are important, is provided on a picture \ref{pic:hard}.

\begin{figure}[H]
\caption{Graph $H$}
\label{pic:hard}
\centering
\includegraphics[width=0.4\textwidth]{pic.2}
\end{figure}

After fixing all mistakes, we evaluated the algorithms using $10^4$ program instances processing $1000$ random graphs with $20$ vertices (without using the algorithm \ref{alg:brute}) and then processing $100$ graphs with $30$ vertices (without using algorithms \ref{alg:brute}, \ref{alg:n6}). The final evaluation did not show any discrepancy between the implemented algorithm results.
