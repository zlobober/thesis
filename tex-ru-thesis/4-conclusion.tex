\chapter{Заключение}

Подведём итог проделанной работе. Взяв за основу модельный алгоритм \ref{alg:main} авторства Tali Eilam-Tzoreff, мы оптимизировали его до времени работы $\OO(|V|^6)$ для невзвешенных графов и до времени работы $\OO(|V|^7)$ для взвешенных графов ценой повышения потребления памяти с $\OO(|V|^4)$ до $\OO(|V|^5)$. 

Также нами была проделана обширная работа по тестированию не только полученных оптимизированных версий алгоритма, но и исходного алгоритма на корректность. Принимая во внимание, что теорема \ref{main_theorem}, взятая нами за основу, весьма непроста как по своей формулировке, так и по характеру своего доказательства \cite{ET}, можно утверждать, что экспериментальное тестирование модельного алгоритма \ref{alg:main} представляет определённую научную пользу. 

Экспериментальную реализацию полученных алгоритмов в виде программ можно расценивать как дополнительный Proof of Concept (доказательство правильности концепции).

Отметим большой простор для дальнейших исследований по теме задачи поиска непересекающихся кратчайших путей. Во-первых, можно заниматься дальнейшим улучшением времени работы алгоритма, решающего задачу $2DSP$ для неориентированного невзвешенного графа. Хорошим свидетельством к тому, что возможно и дальше повысить эффективность решения, является факт того, что самая трудозатратная часть исходного алгоритма потенциально оптимизируется до времени работы $\OO(|V|^{3 + \omega})$ с использованием матричного умножения (оставляя, однако, открытым вопрос относительно оптимизации остальных частей алгоритма до времени, лежащего в $o(|V|^6)$).

Далее остаются открытыми вопросы относительно существования полиномиального алгоритма для решения задачи ориентированной задачи $2DSP$ (взвешенной либо невзвешенной), а также относительно существования полиномиального алгоритма для решения задачи $kDSP$ для $k \geq 3$ (в любой постановке).

Ещё можно заметить, что приведённые оценки сложности выражены только через число вершин в графе. Однако, путём более тщательного анализа сложности алгоритма, скорее всего, можно точнее оценить число операций в полученных алгоритмах, выразив его через число вершин в графе $|V|$ и число рёбер в графе $|E|$, получив тем самым более точную оценку времени работы для разреженных графов.
