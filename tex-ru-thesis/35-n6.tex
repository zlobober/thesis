\chapter{Оптимизация времени работы до $\OO(|V|^6)$}

Можно видеть, что в алгоритме \ref{alg:n7} самым трудозатратным местом стала процедура $\textsc{CalculateAllExYVValues}$, суммарное время которой по всем вызовам составляет $\OO(|V|^7)$. В данной главе мы оптимизируем время на построение значений предиката $ExYV$ на всех значениях аргуметнов до $O(|V|^6)$.

Видоизменим формулу \eqref{eq:ExYV}, переписав её в терминах взятия значения некоторой билинейной формы на наборе векторов. Введём вспомогательное определение:

\begin{definition}
Пусть $V = \{v_0, v_1, \ldots, v_{n-1}\}$. Тогда характеристическим вектором $\chi[S]$ подмножества вершин $S \subseteq V$ назовём такой вектор-столбец, у которого на $i$-й позиции стоит единица, если $v_i \in S$, и ноль в противном случае в некотором фиксированном базисе в $\mathrm{R}^|V|$.
\end{definition}

Наша ближайшая цель --- оформить формулу \eqref{eq:ExYV} в виде подстановки пары векторов $\chi[F(t_1, x, t_2)]$ и $\chi[F(t_2, u, s_1)]$ в некоторую билинейную форму и последующей проверки, равно ли полученное выражение нулю или нет. Остаётся только формально определить билинейную форму, о которой идёт речь. Мы сделаем это, явно выписав матрицу билинейной формы в том же базисе, который использовался в предыдущем определении. Отдельно аккуратно проследим за тем, что в компонентах матрицы нашей формы не оказались значения предиката $2DSP$, которые к данному моменту ещё не вычислены.

\begin{definition}
Пусть к данному моменту определены все значения $DSP(x, y, u, v)$ для $(x, y, u, v)$, таких, что $\min\{l(x, y), l(u, v)\} < d$. Тогда определим матрицу $2DSP(d,x,u) = (2DSP(d,x,u)_{i,j})_{i,j=0,\ldots,n-1}$ следующей формулой:

\begin{equation}
2DSP(d,x,u)_{i,j} = \begin{cases}
    2DSP(x,v_i,u,v_j) & \text{если } l(x,v_i) < d \wedge l(u,v_j) < d \\
    0                 & \text{иначе}
\end{cases}
\end{equation}

\end{definition}

Перепишем с использованием введённых объектов формулу \eqref{eq:ExYV}.

\begin{proposition}
Пусть к данному моменту определены все значения $DSP(x, y, u, v)$ для $(x, y, u, v)$, таких, что $\min\{l(x, y), l(u, v)\} < d$. Пусть также $l(x, t_1) = l(u, t_2) = d - 1$. Тогда верно следующее соотношение:
\begin{equation}
ExYV(x, u, s_1, t_1, t_2) = \chi[F(t_1, x, t_2)]^T~2DSP(d,x,u)~\chi[F(t_2, u, s_1)] > 0 \label{eq:ExYVbil}
\end{equation}
где выражение $\cdot > 0$ следует понимать как индиактор того, что величина больше нуля.
\end{proposition}
\begin{proof}
Пусть существуют такие $y$ и $v$, которые, будучи подставленными вместо соответствующих компонент в правую часть формулы \eqref{eq:ExYV}, доставляют единицу в качестве значения предиката. Пусть эти $y$ и $v$ это вершины $v_i$ и $v_j$ нашего графа. Тогда заметим, что $i$-я компонента $\chi[F(t_1, x, t_2)]$ и $j$-я компонента $\chi[F(t_2, u, s_1)]$ --- единицы в силу того, что $y \in F(t_1, x, t_2)$ и $v \in F(t_2, u, s_1)$ по условиям формулы \eqref{eq:ExYV}, а также, что $2DSP(d, x, u)_{i,j} = 2DSP(x, y, u, v) = 1$ (в силу того, что $l(x, y) = l(u, v) = d - 2 < d$. Значит, значение билинейной формы в правой части \eqref{eq:ExYVbil} по меньшей мере $1$ (так как все компоненты обоих векторов и самой билинейной формы неотрицательны).

Проводя аналогичные рассуждения в обратную сторону, легко видеть, что из положительности значения билинейной формы также следует равенство $ExYV(x, u, s_1, t_1, t_2)$ единице.
\end{proof}

Пока мы не добились никакого принципиального улучшения, так вычисление значения произвольной билинейной формы на заданной паре векторов --- процедура, требующая те же $\Theta(n^2)$ вычислительных действий, где размерность пространства $n$ в нашем случае совпадает с $|V|$. Однако оказывается, что можно ускорить подстановку всех пар векторов в нужные билинейные формы если заметить, что билинейных форм и пар (вектор, форма для подстановки в качестве правого аргумента) не очень много. Формализуем наши наблюдения:

\begin{proposition}
Всевозможных билинейных форм, которые фигурирют в ходе алгоритма, $\OO(|V|^3)$ штук.
\end{proposition}
\begin{proof}
Напрямую следует из того, что каждая билинейная форма задаётся тремя параметрами $d$, $x$ и $u$, каждый из которых принимает в точности $|V|$ значений.
\end{proof}

\begin{proposition}
Всевозможных пар, составленных из билинейной формы и вектора, который оказывается в какой-то момент правым аргументом квадратичной формы, --- $\OO(|V|^4)$.
\end{proposition}
\begin{proof}
Описанные пары имеют вид $(2DSP(d, x, u), \chi[F(t_2, u, s_1)])$ с дополнительным условием, что $l(u, t_2) = d - 1$. Таким образом, всевозможных пар не больше, чем всевозможных наборов $(x, u, t_2, s_1)$, которых в точности $|V|^4$ штук.
\end{proof}

Введём вспомогательную величину:

\begin{equation}
\psi(x, u, t_2, s_1) = 2DSP(l(u, t_2) + 1, x, u)~\chi[F(t_2, u, s_1)] \label{eq:psi}
\end{equation}

Заметим, что каждый вектор $\psi(x, u, t_2, s_1)$ в отдельности возможно вычислить за $\OO(|V|^2)$, значит время, затраченное на вычисление всевозможных $\psi(x, u, t_2, s_1)$ можно оценить как $\OO(|V|^6)$. 

Формула \eqref{eq:ExYVbil} принимает следующий вид:

\begin{equation}
ExYV(x, u, s_1, t_1, t_2) = \chi[F(t_1, x, t_2)]^T \psi(x, u, t_2, s_1) \label{eq:ExYVpsi}
\end{equation}

По полученной формуле отдельное значение $ExYV(x, u, s_1, t_1, t_2)$ вычисляется за время $\OO(|V|)$, значит время, затраченное на вычисление всевозможных $ExYV(x, u, s_1, t_1, t_2)$ можно также оцениить как $\OO(|V|^6)$.

\begin{algorithm}
\caption{Вычисление всех значений $2DSP(s_1, t_1, s_2, t_2)$ за $O(|V|^6)$} \label{alg:n6}
\begin{algorithmic}[1]
\Procedure{CalculateAll2DSPValues}{$V$, $E$}
\State $l \gets \text{матрица попарных расстояний между вершинами в }G$;
\LineComment{Разобьём пары вершин на группы в соответствии с расстояниями между ними.}
\State $P_i \gets \text{пустой список для всех }i = 0, \ldots, |V|-1$;
\For{$s,t \in V$}
    \State $\text{добавляем }(s, t)\text{ в конец списка }P_{l(s, t)}$;
\EndFor
\LineComment{Вычислим значения предиката в порядке лексикографического возрастания}
\LineComment{пары $(d_{\min}, d_{\max})$, где $d_{\min}$ это меньшее из расстояний между парами терминалов,}
\LineComment{а $d_{\max}$ это большее.}
\For{$d_{\min} \gets 0, \ldots, |V|-1$}
    \State $\textsc{CalculateAllExYVValues}(d_{\min})$; \label{line:calcExYVpsi}
    \For{$d_{\max} \gets d_{\min}, \ldots, |V|-1$}
        \For{$(s_1, t_1) \in P_{d_{\min}}$}
            \For{$(s_2, t_2) \in P_{d_{\max}}$}
                \State $2DSP(s_1, t_1, s_2, t_2) \gets \textsc{CalculateSingle2DSPValue}(s_1, t_1, s_2, t_2)$;
                \State $2DSP(s_2, t_2, s_1, t_1) \gets 2DSP(s_1, t_1, s_2, t_2)$; \Comment{по симметрии}
            \EndFor
        \EndFor
    \EndFor
\EndFor
\EndProcedure
\Statex
\Procedure{CalculateAllExYVValues}{$d$}
    \For{$(u, t_2) \in P_{d-1}$}
        \For{$x, s_1 \in V$}
            \State $\psi(x, u, t_2, s_1) \gets 2DSP(l(u, t_2) + 1, x, u)~\chi[F(t_2, u, s_1)]; \label{line:psi}$
        \EndFor
    \EndFor
    \For{$(s_1,t_1 \in P_d)$}
        \For{$x \in F(s_1, t_1)$}
            \For{$u,t_2 \in V$}
                \State $ExYV(x, u, s_1, t_1, t_2) \gets \chi[F(t_1, x, t_2)]^T \psi(x, u, t_2, s_1)$; \label{line:ExYVpsi}
            \EndFor
        \EndFor
    \EndFor
\EndProcedure
\Statex
\Procedure{CalculateSingle2DSPValue}{$s_1$, $t_1$, $s_2$, $t_2$}
\If {$s_1 = t_1 \wedge s_2 = t_2$}
    \State \Return $s_1 = s_2$; 
\ElsIf {$(s_1, t_1, s_2, t_2)\text{ не жёсткая четвёрка}$}
    \State \text{вычисляем и возвращаем значение в соответствии с пунктом б теоремы \ref{main_theorem}};
\Else
    \State $\text{вычисляем }Q_2(s_1, t_1, s_2, t_2)\text{ по формулам \eqref{eq:Q2}}$; 
    \State $Q_4(s_1, t_1, s_2, t_2) = ExYV(F(s_1, t_1, s_2), F(s_2, t_2, t_1), s_1, t_1, t_2)\enspace\vee\enspace$ 
    \Statex $\hspace{4.5cm}ExYV(F(s_2, t_2, s_1), F(s_1, t_1, t_2), s_2, t_2, t_1)$; \label{line:useExYV}
    \State \Return $Q_2(s_1, t_1, s_2, t_2) \vee Q_4(s_1, t_1, s_2, t_2)$;
\EndIf
\EndProcedure
\end{algorithmic}
\end{algorithm}

Подобно предыдущей главе, предъявим конкретный алгоритм в виде псевдокода (алгоритм \ref{alg:n6}) и дадим оценку на время работы и потребляемую память.

\begin{proposition}
Суммарное время работы $\textsc{CalculateAllExYVValues}$ на всех значениях $d = 0, \ldots, |V|-1$ составляет $\OO(|V|^6)$. 
\end{proposition}
\begin{proof}
В первом цикле каждая пара $u, t_2$ будет обработана только при одном значении $d$, совпадающем с $l(u, t_2) + 1$. Значит, суммарное время работы данной процедуры не превосходит количества всевозможных наборов $(u, t_2, x, s_1,)$, помноженного на время $O(|V|^2)$ исполнения строки \ref{line:psi}, что даёт суммарно по всем вызовам время работы $\OO(|V|^6)$.

Во втором цикле каждая пара $s_1, t_1$ будет обработана только при одном значении $d$, совпадающем с $l(s_1, t_1)$. Значит, суммарное время работы данной процедуры не превосходит количества всевозможных наборов $(s_1, t_1, x, u, t_2)$, помноженного на время $O(|V|)$ исполнения строки \ref{line:ExYVpsi}, что даёт суммарно по всем вызовам время работы $\OO(|V|^6)$.
\end{proof}

\begin{proposition}
Объём памяти, потребляемой алгоритмом \ref{alg:n6}, составялет $\OO(|V|^5)$.
\end{proposition}
\begin{proof}
Ко всем значениям, используемым предыдщим алгоритмом \ref{alg:n7}, также добавился четырёхмерной массив векторов длины $n = |V|$, на хранение которого нужно $\OO(|V|^5)$ памяти.
\end{proof}

Оформим полученный результат в виде теоремы:
\begin{theorem}
Алгоритм \ref{alg:n6} находит значения предиката $2DSP(s_1, t_1, s_2, t_2)$ на всех наборах $(s_1, t_1, s_2, t_2)$, используя $O(|V|^6)$ времени и $O(|V|^5)$ памяти. 
\end{theorem}

