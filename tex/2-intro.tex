\Introduction

Поиск комплектов непересекающихся путей между несколькими парами вершин в графах --- широко изученный раздел математики. Бла-бла-бла.

В данной работе изучается задача поиска $k$ непересекающихся кратчайших путей ($k$DSP) в её частном случае, когда $k$ фиксировано и равно двум. Задача поиска $k$ непересекающихся кратчайших путей формулируется следующим образом:

\emph{Дан граф $G = (V, E)$ (ориентированный либо неориентированный, взвешенный либо невзвешенный). Также в графе $G$ выделено $k$ пар терминалов $(s_i, t_i)$ ($s_i, t_i \in V$, $1 \leq i \leq k$). Определить, существуют ли $k$ путей $P_i$, где путь $P_i$ следует из вершины $s_i$ в вершину $t_i$, таких, что 1) каждый путь $P_i$ является одним из кратчайших путей между $s_i$ и $t_i$ и 2) никакие два пути не пересекаются (по рёбрам либо по вершинам).}

Как было показано в работе \cite{ET}, если $k$ является частью входных данных задачи, $k$DSP является NP-полной задачей даже для планарных графов с единичными длинами рёбер вне зависимости от того ориентированный ли граф или нет, и от того, рёберно-пересекающиеся пути мы рассматриваем или вершинно-непересекающиеся. Тем не менее, это не мешает нам рассматривать задачу $k$DSP при фиксированных значениях параметра $k$ в надежде решить её за полиномиальное время. 

В работе \cite{ET} приведён полиномиальный алгоритм, основанный на методе динамического программирования, для решения взвешенной неориентированной задачи $2$DSP для вершинно-непересекающихся путей, а также было приведено сведение рёберно-непересекающегося случая к вершинно-непересекающемуся за линейное от размера графа время. Однако приведённый автором алгоритм обладает весьма неоптимальной временной сложностью $\OO(|V|^8)$, что наводит на мысль о возможности получить более эффективное решение.

Целью данной работы является получение алгоритма с временной сложностью $\OO(|V|^6)$ для решения невзвешенной задачи $2DSP$, а также алгоритма с временной сложностью $\OO(|V|^7)$ для решения взвешенной задачи $2DSP$ (в обоих случаях речь идёт про вершинно-непересекающийся неориентированный вариант задачи). Алгоритмы, полученные в данной статье, основываются на алгоритме, приведённом в \cite{ET}, и являются его последовательными улучшениями с использованием двух идей, первая из которых является стандартной при оптимизации алгоритма с использованием техники динамического программирования, а вторая является оригинальной, не встречавшейся автору в иных статьях. Первая используемая идея заключается в выборе подходящего порядка вычисления значений, позволяющего разделить задачу на две меньшие, решающиеся независимо с уменьшением временной сложности. Вторая идея заключается в оптимизации самого узкого места в алгоритме путём представления подзадачи как взятия значения некоторой билинейной формы на паре векторов, и последующей оценки количества различных возникающих форм и аргументов, которая позволяет сократить вычислительную сложность процедуры подстановки до линейной путём предварительного домножения всех правых аргументов билинейной формы на матрицу формы в некотором фиксированном базисе.

Приведённые алгоритмы были реализованы автором в виде программы на языке программирования C++, с помощью которой корректность алгоритма для невзвешенного случая была протестирована на всевозможных связных графах малого размера ($|V| \leq 8$), а также на значительном количестве случайных связных графов большего размера ($|V| = 10, 20, 30$).

Работа структурирована следующим образом. 

Глава 1 содержит все необходимые определения из статьи \cite{ET}, а также основной результат статьи, на котором основывается б\'{о}льшая часть данной работы. 

Глава 2 содержит ряд вспомогательных утверждений, необходимых для построения модельного алгоритма для невзвешенного случая, с которым и будет вестись дальнейшая работа по оптимизации.

Глава 3 содержит модельный алгоритм для невзвешенного случая с доказательством нижней и верхней оценки на время работы алгоритма.

Глава 4 содержит наблюдения, необходимые, для уменьшения времени работы алгоритма для невзвешеннй задачи в худшем случае до $\OO(|V|^7)$.

Глава 5 содержит наблюдение, необходимые, для уменьшения времени работы алгоритма для невзвешенной задачи в худшем случае до $\OO(|V|^6)$.

Глава 6 содержит практические результаты, полученные во время реализации модельного алгоритма для невзвешенной задачи и его оптимизированных версий в виде программы на языке программирования C++.

Глава 7 содержит ряд дополнительных наблюдений, обобщающих полученный алгоритм на случай взвешенных графов ценой дополнительного множителя в $|V|$ во времени работы, а также потенциально помогающих получить ещё более эффективное решение. 

Исходный код реализации всех полученных алгоритмов, а также tex-исходники данной работы доступны по адресу \url{https://github.com/zlobober/thesis}.
