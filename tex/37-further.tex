\chapter{Обощение на взвешенный случай и дальнейшие наблюдения}

Обобщим алгоритм \ref{alg:n7} на случай взвешенного графа $(G, E)$ с положительной весовой функцией $w : E \rightarrow \mathrm{R}_+$. Сделать это несложно, так как и в самом алгоритме и в его доказательстве расстояния между вершинами только сравниваются между собой, а факт того, что они принимают только значения от $0$ до $|V| - 1$ практически не используется (в отличие от алгоритма \ref{alg:n6}, где расстояние между парой вершин является параметром у билинейной формы).

Предоставим псевдокод алгоритма \ref{alg:n7w}, решающий задачу $2DSP$ для взвешенного графа по-прежнему за время $O(|V|^7)$. 

\begin{algorithm}
\caption{Вычисление всех значений $2DSP(s_1, t_1, s_2, t_2)$ для взвешенного графа за $O(|V|^7)$} \label{alg:n7w}
\begin{algorithmic}[1]
\Procedure{CalculateAll2DSPValues}{$V$, $E$, $w$}
\State $l \gets \text{матрица попарных расстояний между вершинами в }G$;
\State Пусть все уникальные расстояния между парами вершин в порядке сортировки --- $d_0, d_1, \ldots, d_{f-1}$;
\LineComment{Разобьём пары вершин на группы в соответствии с расстояниями между ними.}
\State $P_d \gets \text{пустой список для всех }d = d_0, \ldots, d_{f-1}$;
\For{$s,t \in V$}
    \State $\text{добавляем }(s, t)\text{ в конец списка }P_{l(s, t)}$;
\EndFor
\LineComment{Вычислим значения предиката в порядке лексикографического возрастания}
\LineComment{пары $(d_{\min}, d_{\max})$, где $d_{\min}$ это меньшее из расстояний между парами терминалов,}
\LineComment{а $d_{\max}$ это большее.}
\For{$d_{\min} \gets d_0, \ldots, d_{f-1}$}
    \State $\textsc{CalculateAllExYVValues}(d_{\min})$; \label{line:calcExYVw}
    \For{$d_{\max} \gets d_{\min}, \ldots, d_{f-1}$}
        \For{$(s_1, t_1) \in P_{d_{\min}}$}
            \For{$(s_2, t_2) \in P_{d_{\max}}$}
                \State $2DSP(s_1, t_1, s_2, t_2) \gets \textsc{CalculateSingle2DSPValue}(s_1, t_1, s_2, t_2)$;
                \State $2DSP(s_2, t_2, s_1, t_1) \gets 2DSP(s_1, t_1, s_2, t_2)$; \Comment{по симметрии}
            \EndFor
        \EndFor
    \EndFor
\EndFor
\EndProcedure
\Statex
\Procedure{CalculateAllExYVValues}{$d$}
    \For{$(s_1,t_1 \in P_d)$}
        \For{$x \in F(s_1, t_1)$}
            \For{$u,t_2 \in V$}
                \State $ExYV(x,u,s_1,t_1,t_2) \gets 2DSP(x, F(t_1,x,t_2), u, F(t_2,u,s_1))$;
            \EndFor
        \EndFor
    \EndFor
\EndProcedure
\Statex
\Procedure{CalculateSingle2DSPValue}{$s_1$, $t_1$, $s_2$, $t_2$}
\If {$s_1 = t_1 \wedge s_2 = t_2$}
    \State \Return $s_1 = s_2$; 
\ElsIf {$(s_1, t_1, s_2, t_2)\text{ не жёсткая четвёрка}$}
    \State \text{вычисляем и возвращаем значение в соответствии с пунктом б теоремы \ref{main_theorem}};
\Else
    \State $\text{вычисляем }Q_2(s_1, t_1, s_2, t_2)\text{ по формулам \eqref{eq:Q2}}$; 
    \State $Q_4(s_1, t_1, s_2, t_2) = ExYV(F(s_1, t_1, s_2), F(s_2, t_2, t_1), s_1, t_1, t_2)\enspace\vee\enspace$ 
    \Statex $\hspace{4.5cm}ExYV(F(s_2, t_2, s_1), F(s_1, t_1, t_2), s_2, t_2, t_1)$;
    \State \Return $Q_2(s_1, t_1, s_2, t_2) \vee Q_4(s_1, t_1, s_2, t_2)$;
\EndIf
\EndProcedure
\end{algorithmic}
\end{algorithm}

Заметим, что предложения \ref{n7_corr2}, \ref{n7_rt1}, \ref{n7_rt2} и \ref{n7_m} верны без какой-либо модификации, так как нигде не опираются на факт того, что граф невзвешенный. Предложение \ref{n7_corr1} также остаётся корректным, если заметить, что $l(x, y) < l(s_1, t_1)$ и $l(u, v) < l(s_2, t_2)$ вне зависимости от взвешенности графа.

Таким образом, можно оформить результат в виде теоремы.

\begin{theorem}
Алгоритм \ref{alg:n7w} находит значения предиката $2DSP(s_1, t_1, s_2, t_2)$ для графа $G = (V, E)$ с положительной весовой функцией $w$ на рёбрах на всех наборах $(s_1, t_1, s_2, t_2)$, используя $\OO(|V|^7)$ времени и $\OO(|V|^5)$ памяти. 
\end{theorem}

Отметим также напоследок одну интересную особенность полученного алгоритма \ref{alg:n6}, решающего задачу для невзвешенного случая за время $\OO(|V|^6)$. 

Заметим, что в строке \ref{line:psi} одна и та же матрица умножается на большое количество векторов справа. Если сгруппировать для каждой матрицы все $|V|$ вектор-столбцов (соответствующих выбору вершины $s_1$), на неё умножающихся, в единую матрицу, то описанная процедура примет вид матричного умножения, которое можно делать значительно более эффективно за время $\OO(|V|^\omega)$, где $\omega < 2.3727$ (\cite{Williams}). Однако даже если мы воспользуемся данным подходом и оптимизируем до $\OO(|V|^{3+\omega})$ время вычисления всех значений, соответствующих части $Q_4$ оригинальной теоремы, у нас по-прежнему останутся вычисления, связанный с частью $Q_2$ теоремы, которые следует также ускорить по сравнению со временем $\OO(|V|^6)$.


