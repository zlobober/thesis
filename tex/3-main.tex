\chapter{Основной результат Eilam-Tzoreff}

Начнём с формулировки основного результата работы \cite{ET} в отношении неориентированного вершинно-непересекающегося варианта задачи 2DSP. Введём ряд вспомогательных определений. Здесь и далее $G = (V, E)$ --- неориентированный невзвешенный (с единичными длинами рёбер) граф без петель.

\begin{definition} 
Пусть $x, y \in V$ --- две вершины, находящиеся в одной компоненте связности. Тогда $l(x, y)$ это длина кратчайшего пути из $x$ в $y$ (в рёбрах). Для вершин $x$ и $y$ из разных компонент связности будем полагать значение $l(x, y)$ равным бесконечности.
\end{definition}

В частности, для любой вершины $x \in V$ верно соотношение $l(x, x) = 0$.

\begin{definition} \label{is_between}
Пусть $x, y \in V$ --- две вершины графа. Тогда $L(x, y) \subseteq V$ --- это множество всех вершин $v$, лежащих на хотя бы одном кратчайшем пути из $x$ в $y$.
\end{definition}

Легко видеть, что $v \in L(x, y) \Leftrightarrow l(x, v) + l(v, y) = l(x, y)$. Также отметим, что для несвязных вершин $x$ и $y$ верно соотношение $L(x, y) = \varnothing$.

\begin{definition}
Пусть $x, y \in V$ --- две вершины графа. Тогда 
\begin{equation}
F(x, y) = \{v \in L(x, y) \mid (x, v) \in E\}.
\end{equation}
Иными словами, $F(x, y)$ --- множество всевозможных вершин, следующих за $x$ на каком-либо кратчайшем пути из $x$ в $y$. 
\end{definition}

Для нашего удобства нам также пригодится аналогичное обозначение для множества вершин, следующих на $x$ одновременно и на каком-то кратчайшем пути в $y$, и на каком-то кратчайшем пути в $z$. Введём альтернативный вариант функции $F$, принимающий три аргумента, а не два:

\begin{definition}
Пусть $x, y, z \in V$ --- три вершины графа. Тогда
\begin{equation}
F(x, y, z) = F(x, y) \cap F(x, z)
\end{equation}
\end{definition}

\begin{definition}
Пусть $s_1, t_1, s_2, t_2 \in V$ --- четыре вершины графа. Тогда обозначим за $2DSP(s_1, t_1, s_2, t_2)$ логическую величину, равную $1$, если существуют два вершинно-непересекающихся кратчайших пути $P_1$ и $P_2$ между парами терминалов $(s_1, t_1)$ и $(s_2, t_2)$ соответственно, и $0$ в противном случае.
\end{definition}

\begin{definition}
Будем называть четвёрку вершин $(s_1, t_1, s_2, t_2)$ \textit{жёсткой}, если $s_1, t_1 \in L(s_2, t_2)$ и $s_2, t_2 \in L(s_1, t_1)$.
\end{definition}

Будем придерживаться договорённости, что вместо какого-то из аргументов предиката $2DSP$ может находиться не вершина, а множество вершин. В таком случае значение выражения будем считать логическую дизъюнкцию значений предиката $2DSP$ по всевозможным наборам значений, где на позиции, на которой в записи находится множество вершин, стоит какая-то из вершин указанного множества. Например, в соответствии с такой договорённостью верно соотношение: 

\begin{equation}
2DSP(F(s_1, t_1), t_1, s_2, t_2) = \bigvee\limits_{x \in F(s_1, t_1)} 2DSP(x, t_1, s_2, t_2).
\end{equation}

Теперь мы готовы сформулировать основной результат работы \cite{ET}:

\begin{theorem} \label{main_theorem}
Для $2DSP(s_1, t_1, s_2, t_2)$ верно одно из трёх соотношений:
\begin{enumerate} 
\item Если $s_1 = t_1$, $s_2 = t_2$, то $2DSP(s_1, t_1, s_2, t_2) = 1$ если $s_1 \neq s_2$ и $0$ в противном случае.
\item Иначе, если $(s_1, t_1, s_2, t_2)$ --- нежёсткая четвёрка, то рассмотрим вершину, для которой нарушилось условие жёсткости. Не умаляя общности будем считать, что это $s_1 \notin L(s_2, t_2)$, и при этом $s_1 \neq t_1$. Тогда $2DSP(s_1, t_1, s_2, t_2) = 2DSP(F(s_1, t_1), s_2, t_2)$.
\item Иначе, положим $C = L(s_1, s_2) \cup L(s_2, t_1) \cup L(t_1, t_2) \cup L(t_2, s_1)$. Тогда $2DSP(s_1, t_1, s_2, t_2) = Q_2(s_1, t_1, s_2, t_2) \vee Q_4(s_1, t_1, s_2, t_2)$, где $Q_2$ и $Q_4$ определяются следующим образом.
\begin{align}
Q_2(s_1, t_1, s_2, t_2) =\hspace{-1cm}\\
    2DSP(&& s_1              ,&& F(t_1, s_1, s_2) ,&& s_2              ,&& F(t_2, s_2, s_1) &)~\vee \\
    2DSP(&& F(s_1, t_1, s_2) ,&& t_1              ,&& s_2              ,&& F(t_2, s_2, t_1) &)~\vee \\
    2DSP(&& s_1              ,&& F(t_1, s_1, t_2) ,&& F(s_2, t_2, s_1) ,&& t_2              &)~\vee \\
    2DSP(&& F(s_1, t_1, t_2) ,&& t_1              ,&& F(s_2, t_2, t_1) ,&& t_2              &)~\vee \\[0.4cm]
    %
    2DSP(&& F(s_1, t_1) \sm C,&& t_1              ,&& F(s_2, t_2) \sm C,&& t_2              &)~\vee \\ 
    2DSP(&& F(s_1, t_1) \sm C,&& t_1              ,&& s_2              ,&& F(t_2, s_2) \sm C&)~\vee \\ 
    2DSP(&& s_1              ,&& F(t_1, s_1) \sm C,&& F(s_2, t_2) \sm C,&& t_2              &)~\vee \\ 
    2DSP(&& s_1              ,&& F(t_1, s_1) \sm C,&& s_2              ,&& F(t_2, s_2) \sm C&)~\vee \\[0.4cm]
    %
    2DSP(&& s_1              ,&& F(t_1, s_1, s_2) ,&& s_2              ,&& F(t_2, s_2) \sm C&)~\vee \\
    2DSP(&& s_1              ,&& F(t_1, s_1, t_2) ,&& F(s_2, t_2) \sm C,&& t_2              &)~\vee \\
    2DSP(&& F(t_1, s_1, s_2) ,&& t_1              ,&& s_2              ,&& F(t_2, s_2) \sm C&)~\vee \\
    2DSP(&& F(t_1, s_1, t_2) ,&& t_1              ,&& F(s_2, t_2) \sm C,&& t_2              &)~\vee \\
    2DSP(&& s_1              ,&& F(t_1, s_1) \sm C,&& s_2              ,&& F(t_2, s_2, s_1) &)~\vee \\
    2DSP(&& F(s_1, t_1) \sm C,&& t_1              ,&& s_2              ,&& F(t_2, s_2, t_1) &)~\vee \\
    2DSP(&& s_1              ,&& F(t_1, s_1) \sm C,&& F(s_2, t_2, s_1) ,&& t_2              &)~\vee \\
    2DSP(&& F(s_1, t_1) \sm C,&& t_1              ,&& F(s_2, t_2, t_1) ,&& t_2              &) \label{eq:Q2}
\end{align}
\begin{align}
    Q_4(s_1, t_1, s_2, t_2) = \quad
        \smashoperator{\bigvee_{
        \large \substack{
            x \in F(s_1, s_2)\\ 
            y \in F(t_1, t_2)\\ 
            u \in F(s_2, t_1)\\ 
            v \in F(t_2, s_1)\\ 
            l(x, y) + 2 = l(s_1, t_1)\\ 
            l(u, v) + 2 = l(s_2, t_2)
        }
        }} 2DSP(x, y, u, v)
        \quad\quad\vee\quad\quad
        \smashoperator{\bigvee_{
        \large \substack{
            x \in F(s_1, t_2)\\ 
            y \in F(t_1, s_2)\\ 
            u \in F(s_2, s_1)\\ 
            v \in F(t_2, t_1)\\ 
            l(x, y) + 2 = l(s_1, t_1)\\ 
            l(u, v) + 2 = l(s_2, t_2)
        }
        }} 2DSP(x, y, u, v) \label{eq:Q4} \\
\end{align}
\end{enumerate}
\end{theorem}

У неподготовленного читателя формулировка теоремы выше (а в особенности формулы \eqref{eq:Q2} и \eqref{eq:Q4}) могут вызвать состояние лёгкого шока, поэтому себе остановиться в этом месте и прокомментировать разные аспекты утверждения выше, чтобы по возможности развить у читателя некоторую интуицию относительно выражений, фигурирующих в правой части рекурсивной формулы.

\chapter{Наблюдения относительно структуры основной теоремы}

Пункт а теоремы \ref{main_theorem} не вызывает никаких вопросов и задаёт основание для рекурсивной формулы: действительно, для вырожденного случая, когда пары терминалов схлопываются в одиночные вершины, ответ на вопрос задачи зависит исключительно от того, совпадут ли эти вершины или нет.

Пункт б теоремы \ref{main_theorem} тоже не вызывает особых сложностей. Применим индуктивный аргумент: пусть, например, $s_1 \notin L(s_2, t_2)$, $s_1 \neq s_2$ и при этом $2DSP(x, t_1, s_2, t_2) = 1$ для некоторого $x \in F(s_1, t_1)$. Рассмотрим пару кратчайших путей $(P_1', P_2)$, существующую для двух пар терминалов $(x, t_1)$ и $(s_2, t_2)$. Покажем, как можно расширить первый путь $P_1'$ до пути $P_1$, соединяющего $s_1$ и $t_1$ кратчайшим образом и не пересекающегося с $P_2$ (что докажет, что $2DSP(s_1, t_1, s_2, t_2)$ действительно есть $1$).

Положим $P_1 = \{s_1\} \sqcup P_1'$. Заметим, что $P_1' \cap P_2 = \varnothing$, а $s_1 \notin L(s_2, t_2) \supseteq P_2$. Значит, действительно, $P_1 \cap P_2 = \varnothing$. С другой стороны, если $P_1, P_2$ --- искомая пара кратчайших путей, соединяющих $s_1$ с $t_1$ и $s_2$ с $t_2$, то можно взять в качестве $x$ первую вершину, следующую за $s_1$ в пути $P_1$, и получить таким образом, что $2DSP(x, t_1, s_2, t_2)$, так как $P_1' = P_1 \sm \{s_1\}$ и $P_2$ --- пара кратчайших непересекающихся путей, что доказывает часть б теоремы.

Части в теоремы \ref{main_theorem} описывает те случаи, когда применить индуктивный аргумент, описанный выше, не удаётся. В работе \cite{ET} путём скурпулёзного рассмотрения случаев показывается, что значение $2DSP(s_1, t_1, s_2, t_2)$ можно определить, взглянув на значения предиката $2DSP$ на 18 группах четвёрок вершин, компоненты которых совпадают с соответствующими компонентами четвёрки $(s_1, t_1, s_2, t_2)$, либо смежны с ними определённым образом. Отметим ряд важных свойств формул выше.

\begin{proposition} \label{q2_structure}
В каждом выражении, составляющем $Q_2$, ровно два аргумента совпадают с соответствующими аргументами предиката в левой части, а оставшиеся два аргумента соответствуют множествам вершин, смежных с соответствующими аргументами предиката в левой части. Более того, для каждой пары терминалов $(s_i, t_i)$ ровно один из терминалов остаётся нетронутым, а другой заменяется на множество смежных с ним вершин, находящихся ближе к противоположному терминалу.
\end{proposition}

\begin{proposition}
Оба выражения, составляющие $Q_4$, содержат в качестве каждого из аргументов множество смежных вершин с соответствующим аргументом предиката в левой части. 
\end{proposition}

Назовём четвёрку $(s_1, t_1, s_2, t_2)$ зависящей от четвёрки $(x, y, u, v)$, если в рекурсивной формуле для $2DSP(s_1, t_1, s_2, t_2)$ в правой части встречается значение $2DSP(x, y, u, v)$ (как составная часть одного из выражений либо в $Q_2$, либо в $Q_4$). Будем записывать факт зависимости таких четвёрок как $(s_1, t_1, s_2, t_2) \succ (x, y, u, v)$.

\begin{proposition}
Отношение $\succ$ не содержит циклов. Иными словами, с использованием теоремы \ref{main_theorem} возможно вычислить значения предиката $2DSP$ во всех точках с использованием метода динамического программирования.
\end{proposition}
\begin{proof}
В пункте а теоремы \ref{main_theorem} в правой части формулы для $2DSP(s_1, t_1, s_2, t_2)$ отсутствуют какие-либо другие значения предиката, поэтому такие четвёрки $(s_1, t_1, s_2, t_2)$ ни от чего не зависят.

Теперь заметим такое свойство пунктов б и в теоремы \ref{main_theorem}, что если четвёрка $(s_1, t_1, s_2, t_2) \succ (x, y, u, v)$, то $l(x, y) \leq l(s_1, t_1)$, $l(u, v) \leq l(s_2, t_2)$ и в хотя бы одном из этих двух неравеств знак строгий. В $Q_2$ это верно в силу предложения \ref{q2_structure} о структуре формул, составляющих $Q_2$, а в $Q_4$ это верно по той причине, что всегда выполняются соотношение $l(x, y) = l(s_1, t_1) - 2$ и $l(u, v) = l(s_2, t_2) - 2$.
\end{proof}

Прежде чем записать полностью алгоритм для вычисления всех значений предиката $2DSP$, установим ещё один факт про четвёрки, для которых реализуется случай в теоремы \ref{main_theorem}.

\begin{proposition}
В жёсткой четвёрке $(s_1, t_1, s_2, t_2)$ расстояния между парами терминалов совпадают, то есть, $l(s_1, t_1) = l(s_2, t_2)$.
\end{proposition}
\begin{proof} ~

$s_1, t_1 \in L(s_2, t_2) \Rightarrow l(s_1, s_2) + l(s_2, t_1) = l(s_1, t_2) + l(t_2, t_1)$, т.к. оба выражения равны $l(s_1, t_1)$.

$s_2, t_2 \in L(s_1, t_1) \Rightarrow l(s_2, s_1) + l(s_1, t_2) = l(s_2, t_1) + l(t_1, t_2)$, т.к. оба выражения равны $l(s_2, t_2)$.

Путём сложения двух последних равенств и сокращения одинаковых слагаемых, получим соотношение $l(s_1, s_2) = l(t_2, t_1)$. 

Заменяя $l(s_1, s_2)$ на $l(t_2, t_1)$ в первом равенстве, получим также, что $l(s_2, t_1) = l(s_1, t_2)$.

Наконец, заметим, что $l(s_1, t_1) = l(s_1, s_2) + l(s_2, t_1) = l(t_1, t_2) + l(s_2, t_1) = l(s_2, t_2)$, что завершает доказательство предложения.
\end{proof}

\chapter{Модельный алгоритм}

Установленных фактов хватает, чтобы оформить алгоритм вычисления всех значений предиката $2DSP$ в виде псевдокода (алгоритм \ref{alg:main}). 

\begin{algorithm}
\caption{Вычисление всех значений $2DSP(s_1, t_1, s_2, t_2)$ за $O(|V|^8)$} \label{alg:main}
\begin{algorithmic}[1]
\Procedure{CalculateAll2DSPValues}{$V$, $E$}
\State $l \gets \text{матрица попарных расстояний между вершинами в }G$; \label{line:floyd}
\LineComment{Разобьём пары вершин на группы в соответствии с расстояними между ними.}
\State $P_i \gets \text{пустой список для всех }i = 0, \ldots, |V|-1$;
\For{$s,t \in V$}
    \State $\text{добавляем }(s, t)\text{ в конец списка }P_{l(s, t)}$;
\EndFor
\LineComment{Вычислим значения предиката в порядке лексикографического возрастания}
\LineComment{пары $(d_{\min}, d_{\max})$, где $d_{\min}$ это меньшее из расстояний между парами терминалов,}
\LineComment{а $d_{\max}$ это большее.}
\For{$d_{\min} = 0, \ldots, |V|-1$}
    \For{$d_{\max} = d_{\min}, \ldots, |V|-1$}
        \For{$(s_1, t_1) \in P_{d_{\min}}$}
            \For{$(s_2, t_2) \in P_{d_{\max}}$}
                \State $2DSP(s_1, t_1, s_2, t_2) = \textsc{CalculateSingle2DSPValue}(s_1, t_1, s_2, t_2)$;
                \State $2DSP(s_2, t_2, s_1, t_1) \gets 2DSP(s_1, t_1, s_2, t_2)$; \Comment{по симметрии}
            \EndFor
        \EndFor
    \EndFor
\EndFor
\EndProcedure
\Statex
\Procedure{CalculateSingle2DSPValue}{$s_1$, $t_1$, $s_2$, $t_2$}
\If {$s_1 = t_1 \wedge s_2 = t_2$}
    \State \Return $s_1 = s_2$; \label{line:a}
\ElsIf {$(s_1, t_1, s_2, t_2)\text{ не жёсткая четвёрка}$}
    \State \text{вычисляем и возвращаем значение в соответствии с пунктом б теоремы \ref{main_theorem}}; \label{line:b}
\Else
    \State $\text{вычисляем }Q_2(s_1, t_1, s_2, t_2)\text{ по формулам \eqref{eq:Q2}}$; \label{line:cQ2}
    \State $\text{вычисляем }Q_4(s_1, t_1, s_2, t_2)\text{ по формулам \eqref{eq:Q4}}$; \label{line:cQ4}
    \State \Return $Q_2(s_1, t_1, s_2, t_2) \vee Q_4(s_1, t_1, s_2, t_2)$;
\end{algorithmic}
\end{algorithm}

Прокомментируем временную сложность всех составных частей алгоритма.

\begin{proposition}
Построение матрицы кратчайших расстояний в строке \ref{line:floyd} алгоритма \ref{alg:main} можно произвести за время $\OO(|V|^3)$.
\end{proposition}
\begin{proof}
Достаточно воспользоваться алгоритмом Флойда-Уоршелла (\cite{Floyd}, \cite{CLRS}).
\end{proof}

\begin{proposition}
Процедура \textsc{CalculateSingle2DSPValue} работает за $\OO(1)$, $\OO(|V|)$ или $\OO(|V|^4)$ в зависимости от того, какой случай из теоремы \ref{main_theorem} реализуется.
\end{proposition}
\begin{proof}
Легко видеть, что строка \ref{line:a} состоит из единственной инструкции и соответствует случаю, требующему константное количество операций, строка \ref{line:b} соответствует случаю, в котором вычисление зависит от $\OO(|V|)$ четвёрок (см. пункт б теоремы \ref{main_theorem}). 

Строка \ref{line:cQ2} соответствует группе тех четвёрок вершин, от которых зависит четвёрка $(s_1, t_1, s_2, t_2)$, и которые отличаются от нашей четвёрки в двух позициях, стало быть, вычисление $Q_2(s_1, t_1, s_2, t_2)$ требует не более $\OO(|V|^2)$ операций в худшем случае.

Наконец, строка \ref{line:cQ4} соответствует группе тех четвёрок вершин, которые отличаются от четвёрки $(s_1, t_1, s_2, t_2)$ по каждой компоненте, а значит, вычисление $Q_4(s_1, t_1, s_2, t_2)$ требует не более $\OO(|V|^4)$ операций в худшем случае.
\end{proof}

Таким образом, мы получили верхнюю оценку на время работы алгоритма \ref{alg:main}, которую можно сформулировать в виде теоремы:

\begin{theorem}
Время работы алгоритма \ref{alg:main} есть $\OO(|V|^8)$.
\end{theorem}
\begin{proof}
Основное время занимают вызовы \textsc{CalculateSingle2DSPValue}, которых будет произведено ровно $|V|^4$, и каждый из которых работает за время $O(|V|^4)$.
\end{proof}

Видно, что самым узким местом в программе является вычисление $Q_4(s_1, t_1, s_2, t_2)$. Покажем также, что в худшем случае алгоритм \ref{alg:main} работает за $\Theta(|V|^8)$. Для этого достаточно предъявить семейство графов сколь угодно большого размера, для которых время работы алгоритма \ref{alg:main} есть $\Theta(|V|^8)$.

\begin{theorem}
Построим граф $G_k = (V_k, E_k)$, где 

\begin{align}
V_k &= \bigsqcup\limits_{i = 0,\ldots,7} S_i \sqcup \{c\} \\
E_k &= \bigsqcup\limits_{i = 0,\ldots,7} S_i \times S_{(i+1) \bmod 8} \quad\sqcup\quad \bigsqcup\limits_{i=1,3,5,7} S_i \times \{c\} \\
|S_i| &= k.
\end{align}

Заметим, что в данном графе $|V_k| = 12k + 1$ и $|E_k| = 12k^2 + 4k$. Тогда время работы алгоритма \ref{alg:main} на графе $G_k$ составит $\Theta(k^8) = \Theta(|V|^8)$.

\end{theorem}
\begin{proof}
Предположим, $s_1 \in S_0$, $s_2 \in S_2$, $t_1 \in S_4$, $t_2 \in S_6$. Тогда $l(s_1, t_1) = l(s_2, t_2) = 4$, $l(s_1, s_2) = l(s_2, t_1) = l(t_1, t_2) = l(t_2, s_1) = 2$, то есть, $s_1, t_1 \in L(s_2, t_2)$ и $s_2, t_2 \in L(s_1, t_1)$, а значит, $(s_1, t_1, s_2, t_2)$ --- жёсткая четвёрка вершин.

Заметим также, что если $x \in S_1$, $u \in S_3$, $y \in S_5$, $v \in S_7$, то $(s_1, t_1, s_2, t_2) \succ (x, y, u, v)$. Действительно, верны соотношения $x \in F(s_1, s_2)$, $u \in F(s_2, t_1)$, $y \in F(t_1, t_2)$, $v \in F(t_2, s_1)$, и при этом $l(x, y) = 2 = l(s_1, t_1) - 2$ и $l(u, v) = 2 = l(s_2, t_2) - 2$, значит, $(x, y, u, v)$ попадает в первое выражение формулы \ref{eq:Q4}.

Наконец заметим, что описанных наборов $(s_1, t_1, s_2, t_2, x, y, u, v)$ в точности $k^8 = \Theta(|V|^8)$, а значит, суммарное время, затраченное на исполнение строки \ref{line:cQ4} алгоритма \ref{alg:main}, составит $\Theta(|V|^8)$.
\end{proof}
